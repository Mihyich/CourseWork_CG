\chapter*{Заключение}
\addcontentsline{toc}{chapter}{Заключение}

Цель, которая была поставлена в начале лабораторной работы, была достигнута:
изучен метод умножения квадратных матриц
по стандартному алгоритму и алгоритму Копперсмита-Винограда.

Решены все поставленные задачи:

\begin{itemize}
	\item Изучен метод умножения матриц;
	\item Разработаны алгоритмы умножения матриц: стнадартный, Копперсмит-Виноград и оптимизированный Копперсмит-Виноград;
	\item Реализованы разработанные алгоритмы;
	\item Выполнена оценка затрат алгоритмов по памяти;
	\item Выполнены замеры процессорного времени работы реализаций алгоритмов;
\end{itemize}

В результате проведенных экспериментов было определено,
что cамой медленной реализацией оказался стандартный алгоритм умножения матриц.
Наиболее быстрым оказался оптимизированный алгоритм Копперсмита-Винограда.

Несмотря на то, что алгоритм Копперсмита-Винограда выигрывает по скорости на больших
размерностях матриц, при маленьких значениях выигрыш выделяется на сильно.