\begin{figure}
\centering
\begin{tikzpicture}[y=1cm, x=1cm, yscale=\globalscale*0.74,xscale=\globalscale*0.74, every node/.style={font=\scriptsize}, every node/.append style={scale=\globalscale}, inner sep=0pt, outer sep=0pt]
  \path[draw=black,miter limit=10.0,dash pattern=on 0.0794cm off 0.0794cm] (13.1233, 19.9708) -- (13.3879, 19.9708) -- (13.6525, 19.9713);



  \path[draw=black,fill=white,dash pattern=on 0.0794cm off 0.0794cm] (8.89, 22.0813) rectangle (13.1233, 17.8612);



  \path[draw=black,miter limit=10.0,dash pattern=on 0.0794cm off 0.0794cm] (4.2333, 14.6598) -- (4.7625, 14.6465);



  \path[draw=black,fill=white,dash pattern=on 0.0794cm off 0.0794cm] (0.0, 19.9514) rectangle (4.2333, 9.3681);



  \path[draw=black,miter limit=10.0,dash pattern=on 0.0794cm off 0.0794cm] (3.7042, 20.9833) -- (4.7625, 20.9833);



  \path[draw=black,miter limit=10.0] (2.1167, 20.1896) -- (2.1167, 19.6604);



  \path[draw=black,fill=white,rounded corners=0.4445cm] (0.5292, 21.7771) rectangle (3.7042, 20.1896);



  \begin{scope}[shift={(-0.0132, 0.0132)}]
    \node[text=black,anchor=south,fit={(0,0) (3.175, 1.5875)}] (text5442) at (2.1167, 20.8756){Начало};



  \end{scope}
  \path (4.7625, 22.3062) rectangle (5.7944, 19.6604);



  \path[draw=black,miter limit=10.0] (5.7944, 22.3062) -- (4.7625, 22.3062) -- (4.7625, 19.6604) -- (5.7944, 19.6604);



  \begin{scope}[shift={(-0.0132, 0.0132)}]
    \node[text=black,anchor=south,fit={(0,0) (3, 1.5875)},scale=1] (text5883) at (6.2, 20.8756){Алгоритм теневых карт с фильтрацией (PCF)};



  \end{scope}
  \path[draw=black,miter limit=10.0] (2.1167, 18.0729) -- (2.1167, 17.557);



  \path[draw=black,fill=white] (0.5292, 19.6604) rectangle (3.7042, 18.0729);



  \begin{scope}[shift={(-0.0132, 0.0132)}]
    \node[text=black,anchor=south,fit={(0,0) (3.175, 1.5875)}] (text1439) at (2.1167, 18.759){Инициализировать теневую карту};



  \end{scope}
  \path[draw=black,miter limit=10.0] (2.1167, 15.9695) -- (2.1167, 15.4535);



  \path[draw=black,fill=white] (0.5292, 17.557) rectangle (3.7042, 15.9695);



  \begin{scope}[shift={(-0.0132, 0.0132)}]
    \node[text=black,anchor=south,fit={(0,0) (3.175, 1.5875)}] (text4655) at (2.1167, 16.6688){Инициализировать источник света};



  \end{scope}
  \path[draw=black,miter limit=10.0] (2.1167, 13.866) -- (2.1167, 13.3369);



  \path[draw=black,fill=white] (0.5292, 15.4535) rectangle (3.7042, 13.866);



  \begin{scope}[shift={(-0.0132, 0.0132)}]
    \node[text=black,anchor=south,fit={(0,0) (3.175, 1.5875)}] (text9135) at (2.1167, 14.5521){Инициализировать камеру};



  \end{scope}
  \path[draw=black,miter limit=10.0] (2.1167, 11.7494) -- (2.1167, 11.2202);



  \path[draw=black,fill=white] (0.5292, 13.3369) rectangle (3.7042, 11.7494);



  \begin{scope}[shift={(-0.0132, 0.0132)}]
    \node[text=black,anchor=south,fit={(0,0) (3.175, 1.5875)}] (text9597) at (2.1167, 12.4354){Инициализировать модели};



  \end{scope}
  \path (4.7625, 16.1962) rectangle (5.7944, 13.0969);



  \path[draw=black,miter limit=10.0] (5.7944, 16.1962) -- (4.7625, 16.1962) -- (4.7625, 13.0969) -- (5.7944, 13.0969);



  \begin{scope}[shift={(-0.0132, 0.0132)}]
    \node[text=black,anchor=south,fit={(0,0) (3.175, 1.5875)},scale=0.9] (text121) at (6.35, 14.5521){Подготовительный этап, инициализирующий минимально необходимые данные};



  \end{scope}
  \path[draw=black,miter limit=10.0] (2.1167, 9.6327) -- (2.1167, 9.1017) -- (8.0963, 9.1017) -- (8.0963, 22.352) -- (11.0067, 22.352) -- (11.0067, 21.8035);



  \path[draw=black,fill=white] (0.5292, 11.2202) rectangle (3.7042, 9.6327);



  \begin{scope}[shift={(-0.0132, 0.0132)}]
    \node[text=black,anchor=south,fit={(0,0) (3.175, 1.5875)}] (text1075) at (2.1167, 10.3188){Инициализировать
шейдеры};



  \end{scope}
  \path[draw=black,miter limit=10.0] (11.0067, 20.216) -- (11.0067, 19.6869);



  \path[draw=black,fill=white] (9.4192, 21.8035) rectangle (12.5942, 20.216);



  \begin{scope}[shift={(-0.0132, 0.0132)}]
    \node[text=black,anchor=south,fit={(0,0) (3.175, 1.5875)}] (text8288) at (11.0067, 20.9021){Нарисовать сцену с точки зрения источника света};



  \end{scope}
  \path[draw=black,miter limit=10.0] (11.0067, 18.0994) -- (11.0067, 17.3321);



  \path[draw=black,fill=white] (9.4192, 19.6869) rectangle (12.5942, 18.0994);



  \begin{scope}[shift={(-0.0132, 0.0132)}]
    \node[text=black,anchor=south,fit={(0,0) (3.175, 1.5875)}] (text6434) at (11.0067, 18.7854){Заполнить теневую карту значениями глубины};



  \end{scope}
  \path (13.6525, 21.2942) rectangle (14.6844, 18.6484);



  \path[draw=black,miter limit=10.0] (14.6844, 21.2942) -- (13.6525, 21.2942) -- (13.6525, 18.6484) -- (14.6844, 18.6484);



  \begin{scope}[shift={(-0.0132, 0.0132)}]
    \node[text=black,anchor=south,fit={(0,0) (2, 1.5875)} ] (text9891) at (14.75, 19.8702){Этап заполнения теневой карты};



  \end{scope}
  \path[draw=black,miter limit=10.0] (11.0067, 16.0091) -- (11.0067, 15.48);



  \path[draw=black,fill=white,miter limit=10.0] (9.9483, 17.3321) -- (12.065, 17.3321) -- (12.5942, 16.9087) -- (12.5942, 16.0091) -- (9.4192, 16.0091) -- (9.4192, 16.9087) -- cycle;



  \begin{scope}[shift={(-0.0132, 0.0132)}]
    \node[text=black,anchor=south,fit={(0,0) (3.175, 1.5875)}] (text5014) at (11.0067, 16.5629){Обработка N пикселей};



  \end{scope}
  \path[draw=black,miter limit=10.0,dash pattern=on 0.0794cm off 0.0794cm] (12.5942, 14.6862) -- (13.3879, 14.6862);



  \path[draw=black,miter limit=10.0] (11.0067, 13.8925) -- (11.0067, 13.3633);



  \path[draw=black,fill=white] (9.4192, 15.48) rectangle (12.5942, 13.8925);



  \begin{scope}[shift={(-0.0132, 0.0132)}]
    \node[text=black,anchor=south,fit={(0,0) (3.175, 1.5875)},scale=0.8] (text8939) at (11.0067, 14.5785){Преобразовать координаты фрагмента в пространство света};



  \end{scope}
  \path (13.3879, 15.4932) rectangle (14.4463, 13.8792);



  \path[draw=black,miter limit=10.0] (14.4463, 15.4932) -- (13.3879, 15.4932) -- (13.3879, 13.8792) -- (14.4463, 13.8792);



  \begin{scope}[shift={(-0.0132, 0.0132)}]
    \node[text=black,anchor=south,fit={(0,0) (3.175, 1.5875)} ] (text9548) at (14.5, 14.5785){Далее
d1};



  \end{scope}
  \path[draw=black,fill=white,rounded corners=0.4445cm] (0.5292, 1.9844) rectangle (3.7042, 0.3969);



  \begin{scope}[shift={(-0.0132, 0.0132)}]
    \node[text=black,anchor=south,fit={(0,0) (3.175, 1.5875)}] (text5062) at (2.1167, 1.0848){Конец};



  \end{scope}
  \path[draw=black,miter limit=10.0] (11.0067, 9.9237) -- (11.0067, 9.3946);



  \path[draw=black,fill=white,miter limit=10.0] (9.9483, 11.2466) -- (12.065, 11.2466) -- (12.5942, 10.8233) -- (12.5942, 9.9237) -- (9.4192, 9.9237) -- (9.4192, 10.8233) -- cycle;



  \begin{scope}[shift={(-0.0132, 0.0132)}]
    \node[text=black,anchor=south,fit={(0,0) (3.175, 1.5875)},scale=0.9] (text1678) at (11.0067, 10.4775){Обработка S блокирующих, окаймляющий пикселей};



  \end{scope}
  \path[draw=black,miter limit=10.0] (11.0067, 11.7758) -- (11.0067, 11.2466);



  \path[draw=black,miter limit=10.0,dash pattern=on 0.0794cm off 0.0794cm] (12.5942, 12.5624) -- (13.3879, 12.5587);



  \path[draw=black,fill=white] (9.4192, 13.3633) rectangle (12.5942, 11.7758);



  \begin{scope}[shift={(-0.0132, 0.0132)}]
    \node[text=black,anchor=south,fit={(0,0) (3.175, 1.5875)}] (text1321) at (11.0067, 12.4619){sumBlocker = 0};



  \end{scope}
  \path[draw=black,miter limit=10.0,dash pattern=on 0.0794cm off 0.0794cm] (12.5942, 8.608) -- (13.3879, 8.6117);



  \path[draw=black,miter limit=10.0] (11.0067, 7.8071) -- (11.0067, 7.276);



  \path[draw=black,fill=white] (9.4192, 9.3946) rectangle (12.5942, 7.8071);



  \begin{scope}[shift={(-0.0132, 0.0132)}]
    \node[text=black,anchor=south,fit={(0,0) (3.175, 1.5875)}] (text1824) at (11.0067, 8.4931){Считать соответствующую глубину в окрестности};



  \end{scope}
  \path (13.3879, 9.421) rectangle (14.4463, 7.8071);



  \path[draw=black,miter limit=10.0] (14.4463, 9.421) -- (13.3879, 9.421) -- (13.3879, 7.8071) -- (14.4463, 7.8071);



  \begin{scope}[shift={(-0.0132, 0.0132)}]
    \node[text=black,anchor=south,fit={(0,0) (3.175, 1.5875)} ] (text5006) at (14.5, 8.5196){Далее
d2};



  \end{scope}
  \path (13.3879, 13.3633) rectangle (14.4463, 11.7494);



  \path[draw=black,miter limit=10.0] (14.4463, 13.3633) -- (13.3879, 13.3633) -- (13.3879, 11.7494) -- (14.4463, 11.7494);



  \begin{scope}[shift={(-0.0132, 0.0132)}]
    \node[text=black,anchor=south,fit={(0,0) (3.175, 1.5875)},scale=0.9] (text3773) at (14.5, 12.4619){Количество блокирующих глубин};



  \end{scope}
  \path[draw=black,miter limit=10.0] (12.5942, 6.4823) -- (13.1022, 6.477) -- (13.1022, 5.3544);



  \path[draw=black,fill=black,miter limit=10.0] (13.1022, 5.2155) -- (13.0096, 5.4007) -- (13.1022, 5.3544) -- (13.1948, 5.4007) -- cycle;



  \begin{scope}[shift={(-0.0132, 0.0132)}]
    \node[text=black,anchor=south,fit={(0,0) (3.175, 1.5875)}] (text4912) at (12.8323, 6.641){Да};



  \end{scope}
  \path[draw=black,miter limit=10.0] (9.4192, 6.4823) -- (8.7842, 6.477) -- (8.7842, 3.0692) -- (11.0067, 3.0692) -- (11.0067, 2.7085);



  \path[draw=black,fill=black,miter limit=10.0] (11.0067, 2.5696) -- (10.9141, 2.7548) -- (11.0067, 2.7085) -- (11.0993, 2.7548) -- cycle;



  \path[draw=black,fill=white,miter limit=10.0] (11.0067, 7.276) -- (12.5942, 6.4823) -- (11.0067, 5.6885) -- (9.4192, 6.4823) -- cycle;



  \begin{scope}[shift={(-0.0132, 0.0132)}]
    \node[text=black,anchor=south,fit={(0,0) (3.175, 1.5875)}] (text3271) at (11.0067, 6.3765){d1 > d2};



  \end{scope}
  \path[draw=black,miter limit=10.0] (13.1022, 3.5983) -- (13.1022, 3.0692) -- (11.0067, 3.0692) -- (11.0067, 2.7085);



  \path[draw=black,fill=black,miter limit=10.0] (11.0067, 2.5696) -- (10.9141, 2.7548) -- (11.0067, 2.7085) -- (11.0993, 2.7548) -- cycle;



  \path[draw=black,fill=white] (11.5094, 5.1858) rectangle (14.6844, 3.5983);



  \begin{scope}[shift={(-0.0132, 0.0132)}]
    \node[text=black,anchor=south,fit={(0,0) (3.175, 1.5875)}] (text3007) at (13.0969, 4.2862){sumBlocker += 1};



  \end{scope}
  \path[draw=black,miter limit=10.0] (11.0067, 1.2171) -- (11.0067, 0.6562) -- (16.0338, 0.6562) -- (16.0338, 22.352) -- (18.6796, 22.352) -- (18.6796, 21.8167);



  \path[draw=black,fill=white,miter limit=10.0,cm={ -1.0,-0.0,0.0,-1.0,(22.0133, 3.7571)}] (9.9483, 2.54) -- (12.065, 2.54) -- (12.5942, 2.1167) -- (12.5942, 1.2171) -- (9.4192, 1.2171) -- (9.4192, 2.1167) -- cycle;



  \begin{scope}[shift={(-0.0132, 0.0132)}]
    \node[text=black,anchor=south,fit={(0,0) (3.175, 1.5875)},scale=0.9] (text4100) at (11.0067, 1.7727){Пока остались  блокирующие пиксели};



  \end{scope}
  \path[draw=black,miter limit=10.0,dash pattern=on 0.0794cm off 0.0794cm] (3.7042, 6.4558) -- (4.7625, 6.4558);



  \path[draw=black,miter limit=10.0] (2.1167, 5.6621) -- (2.1167, 4.6567);



  \path[draw=black,fill=white] (0.5292, 7.2496) rectangle (3.7042, 5.6621);



  \begin{scope}[shift={(-0.0132, 0.0132)}]
    \node[text=black,anchor=south,fit={(0,0) (3.175, 1.5875)}] (text6853) at (2.1167, 6.35){aver /= MxM};



  \end{scope}
  \path (4.7625, 7.2628) rectangle (5.4504, 5.6489);



  \path[draw=black,miter limit=10.0] (5.4504, 7.2628) -- (4.7625, 7.2628) -- (4.7625, 5.6489) -- (5.4504, 5.6489);



  \begin{scope}[shift={(-0.0132, 0.0132)}]
    \node[text=black,anchor=south,fit={(0,0) (3.175, 1.5875)} ] (text9927) at (6.5, 6.35){Среднее арифметическое значение в диапазоне [0, 1]};



  \end{scope}
  \path (4.7625, 4.9213) rectangle (5.4504, 2.7781);



  \path[draw=black,miter limit=10.0] (5.4504, 4.9213) -- (4.7625, 4.9213) -- (4.7625, 2.7781) -- (5.4504, 2.7781);



  \begin{scope}[shift={(-0.0132, 0.0132)}]
    \node[text=black,anchor=south,fit={(0,0) (4.5, 1.5875)},scale=0.65] (text9191) at (6.5, 3.7571){Освещение применяется с учетом значения aver, на которое умножается диффузная и спекулярная состовляющие};



  \end{scope}
  \path[draw=black,miter limit=10.0,dash pattern=on 0.0794cm off 0.0794cm] (3.7042, 3.8629) -- (4.2333, 3.8629) -- (4.7625, 3.8523);



  \path[draw=black,miter limit=10.0] (2.1167, 3.0692) -- (2.1167, 1.9844);



  \path[draw=black,fill=white] (0.5292, 4.6567) rectangle (3.7042, 3.0692);



  \path[draw=black,miter limit=10.0] (0.8467, 4.6567) -- (0.8467, 3.0692)(3.3867, 4.6567) -- (3.3867, 3.0692);



  \begin{scope}[shift={(-0.0132, 0.0132)}]
    \node[text=black,anchor=south,fit={(0,0) (3.175, 1.5875)}] (text9131) at (2.1167, 3.7571){Применить\\полное\\освещение};



  \end{scope}
  \path[draw=black,miter limit=10.0] (18.6796, 3.5983) -- (18.6796, 0.127) -- (8.0963, 0.127) -- (8.0963, 7.7999) -- (2.1167, 7.7999) -- (2.1167, 7.2496);



  \path[draw=black,fill=white,miter limit=10.0,cm={ -1.0,-0.0,0.0,-1.0,(37.3592, 8.5196)}] (17.6212, 4.9213) -- (19.7379, 4.9213) -- (20.2671, 4.4979) -- (20.2671, 3.5983) -- (17.0921, 3.5983) -- (17.0921, 4.4979) -- cycle;



  \begin{scope}[shift={(-0.0132, 0.0132)}]
    \node[text=black,anchor=south,fit={(0,0) (3.175, 1.5875)}] (text5500) at (18.6796, 4.154){Пока остались необработанные пиксели};



  \end{scope}
  \path[draw=black,miter limit=10.0,dash pattern=on 0.0794cm off 0.0794cm] (20.2671, 21.023) -- (20.7963, 21.023);



  \path[draw=black,miter limit=10.0] (18.6796, 20.2292) -- (18.6796, 19.6869);



  \path[draw=black,fill=white] (17.0921, 21.8167) rectangle (20.2671, 20.2292);



  \begin{scope}[shift={(-0.0132, 0.0132)}]
    \node[text=black,anchor=south,fit={(0,0) (3.175, 1.5875)}] (text8144) at (18.6796, 20.9285){bd = \(\frac{sumBlocker}{S}\)};



  \end{scope}
  \path (20.7963, 21.83) rectangle (21.4842, 20.216);



  \path[draw=black,miter limit=10.0] (21.4842, 21.83) -- (20.7963, 21.83) -- (20.7963, 20.216) -- (21.4842, 20.216);



  \begin{scope}[shift={(-0.0132, 0.0132)}]
    \node[text=black,anchor=south,fit={(0,0) (3.175, 1.5875)},scale=0.8] (text8957) at (22, 20.9285){Среднее арифметическое значение в диапазоне [0, 1]};



  \end{scope}
  \path[draw=black,miter limit=10.0,dash pattern=on 0.0794cm off 0.0794cm] (20.2671, 18.8931) -- (20.7963, 18.8931);



  \path[draw=black,miter limit=10.0] (18.6796, 18.0994) -- (18.6796, 17.8329) -- (18.6796, 17.5966);



  \path[draw=black,fill=white] (17.0921, 19.6869) rectangle (20.2671, 18.0994);



  \begin{scope}[shift={(-0.0132, 0.0132)}]
    \node[text=black,anchor=south,fit={(0,0) (3.175, 1.5875)}] (text9770) at (18.6796, 18.7854){$penumbraSize = \frac{d1 - bd}{bd}$};



  \end{scope}
  \path (20.7963, 19.7001) rectangle (21.4842, 18.0861);



  \path[draw=black,miter limit=10.0] (21.4842, 19.7001) -- (20.7963, 19.7001) -- (20.7963, 18.0861) -- (21.4842, 18.0861);



  \begin{scope}[shift={(-0.0132, 0.0132)}]
    \node[text=black,anchor=south,fit={(0,0) (3.175, 1.5875)},scale=0.75] (text7127) at (22, 18.7854){Расчет размера полутени};



  \end{scope}
  \path[draw=black,miter limit=10.0] (18.6796, 14.1552) -- (18.6796, 13.8853) -- (18.6796, 13.6279);



  \path[draw=black,fill=white,miter limit=10.0] (17.6212, 15.4781) -- (19.7379, 15.4781) -- (20.2671, 15.0548) -- (20.2671, 14.1552) -- (17.0921, 14.1552) -- (17.0921, 15.0548) -- cycle;



  \begin{scope}[shift={(-0.0132, 0.0132)}]
    \node[text=black,anchor=south,fit={(0,0) (3.175, 1.5875)},scale=0.9] (text334) at (18.6796, 14.7108){Обработка MxM окаймляющий пикселей};



  \end{scope}
  \path[draw=black,miter limit=10.0] (18.6796, 16.0091) -- (18.6796, 15.4781);



  \path[draw=black,fill=white] (17.0921, 17.5966) rectangle (20.2671, 16.0091);



  \begin{scope}[shift={(-0.0132, 0.0132)}]
    \node[text=black,anchor=south,fit={(0,0) (3.175, 1.5875)}] (text9090) at (18.6796, 16.6952){aver = 0};



  \end{scope}
  \path[draw=black,miter limit=10.0,dash pattern=on 0.0794cm off 0.0794cm] (20.2671, 12.8341) -- (20.7963, 12.8341);



  \path[draw=black,miter limit=10.0] (18.6785, 12.0404) -- (18.6785, 11.5112);



  \path[draw=black,fill=white] (17.0921, 13.6279) rectangle (20.2671, 12.0404);



  \begin{scope}[shift={(-0.0132, 0.0132)}]
    \node[text=black,anchor=south,fit={(0,0) (3.175, 1.5875)}] (text6779) at (18.6796, 12.7265){Считать соответствующую глубину в окрестности};



  \end{scope}
  \path (20.7963, 13.6411) rectangle (21.4842, 12.0272);



  \path[draw=black,miter limit=10.0] (21.4842, 13.6411) -- (20.7963, 13.6411) -- (20.7963, 12.0272) -- (21.4842, 12.0272);



  \begin{scope}[shift={(-0.0132, 0.0132)}]
    \node[text=black,anchor=south,fit={(0,0) (3.175, 1.5875)},scale=0.75] (text1130) at (22, 12.7265){Окрестность масштабируется в penumbraSize раз. Далее d3};



  \end{scope}
  \path[draw=black,miter limit=10.0] (20.266, 10.7175) -- (20.7751, 10.7209) -- (20.7751, 9.5896);



  \path[draw=black,fill=black,miter limit=10.0] (20.7751, 9.4507) -- (20.6825, 9.6359) -- (20.7751, 9.5896) -- (20.8677, 9.6359) -- cycle;



  \begin{scope}[shift={(-0.0132, 0.0132)}]
    \node[text=black,anchor=south,fit={(0,0) (3.175, 1.5875)}] (text198) at (20.5052, 10.8744){Да};



  \end{scope}
  \path[draw=black,miter limit=10.0] (17.091, 10.7175) -- (16.4571, 10.7209) -- (16.4571, 7.3025) -- (18.6796, 7.3025) -- (18.6796, 6.9154);



  \path[draw=black,fill=black,miter limit=10.0] (18.6796, 6.7765) -- (18.587, 6.9617) -- (18.6796, 6.9154) -- (18.7722, 6.9617) -- cycle;



  \path[draw=black,fill=white,miter limit=10.0] (18.6785, 11.5112) -- (20.266, 10.7175) -- (18.6785, 9.9237) -- (17.091, 10.7175) -- cycle;



  \begin{scope}[shift={(-0.0132, 0.0132)}]
    \node[text=black,anchor=south,fit={(0,0) (3.175, 1.5875)}] (text3298) at (18.6796, 10.6098){d1 > d3};



  \end{scope}
  \path[draw=black,miter limit=10.0] (20.7751, 7.8335) -- (20.7751, 7.3025) -- (18.6796, 7.3025) -- (18.6796, 6.9154);



  \path[draw=black,fill=black,miter limit=10.0] (18.6796, 6.7765) -- (18.587, 6.9617) -- (18.6796, 6.9154) -- (18.7722, 6.9617) -- cycle;



  \path[draw=black,fill=white] (19.1812, 9.421) rectangle (22.3562, 7.8335);



  \begin{scope}[shift={(-0.0132, 0.0132)}]
    \node[text=black,anchor=south,fit={(0,0) (3.175, 1.5875)}] (text2209) at (20.7698, 8.5196){aver += 1};



  \end{scope}
  \path[draw=black,miter limit=10.0] (18.6796, 5.424) -- (18.6796, 5.1541) -- (18.6796, 4.9213);



  \path[draw=black,fill=white,miter limit=10.0,cm={ -1.0,-0.0,0.0,-1.0,(37.3592, 12.1708)}] (17.6212, 6.7469) -- (19.7379, 6.7469) -- (20.2671, 6.3235) -- (20.2671, 5.424) -- (17.0921, 5.424) -- (17.0921, 6.3235) -- cycle;



  \begin{scope}[shift={(-0.0132, 0.0132)}]
    \node[text=black,anchor=south,fit={(0,0) (3.175, 1.5875)}] (text1446) at (18.6796, 5.9796){Пока остались  пиксели в окрестности};



  \end{scope}

\end{tikzpicture}

\caption{Схема алгоритма мягких теневых карт с фильтрацией (PCSS)}
\label{chart:shadow_map_pcss}

\end{figure}
