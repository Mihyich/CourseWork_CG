\chapter{Технологический раздел}

В данном разделе описаны используемые языки программирования и среда
разработки.

\section{Выбор языка программирования и среды разработки}

Исследование реализаций алгоритмов будет проведено на операционной системе
\text{«Windows»}. Поэтому для создания оконного приложения используется
интерфейс прикладного программирования Win32 API. Это набор функций
для создания программ, работающих под управлением Microsoft Windows 98,
Windows NT или Windows 2000. Все функции этого набора являются 32-битными,
что отражено в названии интерфейса.~\cite{Win32Api_Shupak}

Инициализация, заполнение, обновление и использование теневых карт
предполагается в шейдерных программах. Для этого требуется инициализация
контекста \textit{«OpenGL»} и загрузка необходимых расширений, 
позволяющих компилировать такие программы. Для того,
чтобы связать контекст и оконное приложение на операционной системе
\text{«Windows»} требуется непосредственно сама библиотека
\textit{«opengl32.lib»}.~\cite{extOpenGL}

В ходе разработки реализации алгоритмов теневых карт потребуется
использование матричных преобразований. Они должны не только позволять
производить непосредственно сами преобразования, но и удовлетворять
требованиям использования в шейдерных программах. А именно их представление
в памяти должно соответсвовать ожиданиям шейдеров. Для этого отдельно
разработан модуль линейной алгебры.~\cite{Linal}

Таким образом, в качестве среды программирования выбрана программа
\text{Visual Studio Code}. Основным языком программирования всего приложения
выбран \text{C++}. Для разработки модуля матричных преобразований выбран язык
\text{C}, который наибольшим образом схож с языком программирования шейдерных
программ -- \text{GLSL}.

\section{Исходные модули программы}

Программа для удобства разделена на модули:

\begin{enumerate}[label=\arabic*), labelsep=0.5em]
    \item модуль \text{«LA»} - статическая библиотека линейной алгебры, реализующая
    матричные преобразования, работу с векторами и кватернионами и обеспечивающая
    выравнивание данных в памяти, ожидаемое в шейдерных программах;
    \item модуль \text{«WINAPI»} - статическая библиотека, обеспечивающая создание
    оконного приложения с поддержкой инициализации контекста \textit{«OpenGL»} (версией 4.6)
    и их совместное связывание;
    \item модуль \text{«GLSL»} - статическая библиотека, предоставляющая компилирование, линкование,
    анализ шейдерных программ, оптимизирует рутинные действия при работе с шейдерами;
    \item модуль \text{«ShadowMap»} - статическая библиотека, в которую собраны исследуемые реализации
    алгоритмов теневых карт;
    \item модуль \text{«app»} является основным модулем всего приложения и связывает все
    модули, упомянутые выше, в единое целое.
\end{enumerate}

Помимо модулей основного приложения далее также будут описаны исходные коды
шейдерных программ.

\section{Исходные файлы модуля \text{«LA»}}

В данной библиотеке реализованы такие математические объекты как:

\begin{itemize}[label=---]
    \item вектор с 2-мя компонентами,
    \item вектор с 3-мя компонентами,
    \item вектор с 4-мя компонентами,
    \item кватернион,
    \item матрица размерностью 2 на 2,
    \item матрица размерностью 3 на 3,
    \item матрица размерностью 4 на 4.
\end{itemize}

Файлы модуля представлены ниже:

\begin{itemize}[label=---]
    \item \textbf{заголовочные} файлы:
    \begin{enumerate}[label=\arabic*), labelsep=0.5em]
        \item \text{LA\_sup.h} -- объявление вспомогательных функций для данного модуля;
        \item \text{Matrix2D.h} -- объявление структуры матрицы размерностью 2 на 2 и функций по ее использованию;
        \item \text{Matrix3D.h} -- объявление структуры матрицы размерностью 3 на 3 и функций по ее использованию;
        \item \text{Matrix4D.h} -- объявление структуры матрицы размерностью 4 на 4 и функций по ее использованию;
        \item \text{Quaternion.h} -- объявление структуры кватерниона и функций по его использованию;
        \item \text{Vector2D.h} -- объявление структуры вектора с 2-мя компонентами и функций по его использованию;
        \item \text{Vector3D.h} -- объявление структуры вектора с 3-мя компонентами и функций по его использованию;
        \item \text{Vector4D.h} -- объявление структуры вектора с 4-мя компонентами и функций по его использованию;
    \end{enumerate}
    \item \textbf{исходные} файлы:
    \begin{enumerate}[label=\arabic*), labelsep=0.5em]
        \item \text{LA\_sup.c} -- реализация вспомогательных функций для данного модуля;
        \item \text{Matrix2D.c} -- реализация матричных функций для размерности 2 на 2;
        \item \text{Matrix3D.c} -- реализация матричных функций для размерности 3 на 3;
        \item \text{Matrix4D.c} -- реализация матричных функций для размерности 4 на 4;
        \item \text{Quaternion.c} -- реализация функций преобразований, задаваемых кватернионом;
        \item \text{Vector2D.c} --  реализация функций взаимодействия с 2-ух компонентными векторами;
        \item \text{Vector3D.c} --  реализация функций взаимодействия с 3-ех компонентными векторами;
        \item \text{Vector4D.c} --  реализация функций взаимодействия с 4-ех компонентными векторами;
    \end{enumerate}
\end{itemize}

\section{Исходные файлы модуля \text{«WINAPI»}}

Файлы модуля представлены ниже:

\begin{itemize}[label=---]
    \item \textbf{заголовочные} файлы:
    \begin{enumerate}[label=\arabic*), labelsep=0.5em]
        \item \text{winapi\_brush\_struct.h} -- объявление структуры кисти;
        \item \text{winapi\_brash.h} -- объявление класса, реализующего инициализацию, использование и освобождение кисти;
        \item \text{winapi\_char\_converter.h} -- объявление функций конвертации строк;
        \item \text{winapi\_choose\_color\_dialog.h} -- объявление функции вызова диалога выбора цвета;
        \item \text{winapi\_choose\_file\_dialog.h} -- объявление функции вызова диалога выбора файла;
        \item \text{winapi\_common.h} -- объявление вспомогательных функций, специфичных для данного модуля;
        \item \text{winapi\_console.h} -- объявление класса, обеспечивающего создание консоли и перенаправления потоков ввода-вывода;
        \item \text{winapi\_font\_common.h} -- объявление общих функций работы с шрифтами;
        \item \text{winapi\_font\_struct.h} -- объявление структуры шрифта;
        \item \text{winapi\_font.h} -- объевление класса, обеспечивающего создание, использование и освобождение шрифта;
        \item \text{winapi\_GLextensions.h} -- объявление функции загрузки расширений контекста \textit{«OpenGL»};
        \item \text{winapi\_GLwindow.h} -- объявление класса, реализующего создание, использование и освобождение окна, поддерживающего связывание с контекстом \textit{«OpenGL»}; 
        \item \text{winapi\_mat\_ext.h} -- объявление общих расчетных функций, специфичных для данного модуля;
        \item \text{winapi\_mouse.h} -- объявление класса мыши, реализующего взаимодействие с вводом мыши;
        \item \text{winapi\_str\_converter.h} -- объявление расширенных функций конвертации строк;
        \item \text{winapi\_window.h} -- объявление класса, реализующего создание, использование и освобождение окна;
    \end{enumerate}
    \item \textbf{исходные} файлы:
    \begin{enumerate}[label=\arabic*), labelsep=0.5em]
        \item \text{winapi\_brash.cpp} -- реализация класса, обеспечивающего инициализацию, использование и освобождение кисти;
        \item \text{winapi\_char\_converter.cpp} -- реализация функций конвертации строк;
        \item \text{winapi\_choose\_color\_dialog.cpp} -- реализация функции вызова диалога выбора цвета;
        \item \text{winapi\_choose\_file\_dialog.cpp} -- реализация функции вызова диалога выбора файла;
        \item \text{winapi\_common.cpp} -- реализация вспомогательных функций, специфичных для данного модуля;
        \item \text{winapi\_console.cpp} -- реализация класса, обеспечивающего создание консоли и перенаправления потоков ввода-вывода;
        \item \text{winapi\_font\_common.cpp} -- реализация общих функций работы с шрифтами;
        \item \text{winapi\_font.cpp} -- реализация класса, обеспечивающего создание, использование и освобождение шрифта;
        \item \text{winapi\_GLextensions.cpp} -- реализация функции загрузки расширений контекста \textit{«OpenGL»};
        \item \text{winapi\_GLwindow.cpp} -- реализация класса, обеспечивающего создание, использование и освобождение окна, поддерживающего связывание с контекстом \textit{«OpenGL»}; 
        \item \text{winapi\_mat\_ext.cpp} -- реализация общих расчетных функций, специфичных для данного модуля;
        \item \text{winapi\_mouse.cpp} -- реализация класса мыши, обеспечивающего взаимодействие с вводом мыши;
        \item \text{winapi\_str\_converter.cpp} -- реализация расширенных функций конвертации строк;
        \item \text{winapi\_window.cpp} -- реализация класса, обеспечивающего создание, использование и освобождение окна;
    \end{enumerate}
\end{itemize}

\section{Исходные файлы модуля \text{«GLSL»}}

Файлы модуля представлены ниже:

\begin{itemize}[label=---]
    \item \textbf{заголовочные} файлы:
    \begin{enumerate}[label=\arabic*), labelsep=0.5em]
        \item \text{shader\_extensions.h} -- объявление функций упрощения отправки данных в шейдерные программы;
        \item \text{shader.h} -- объявление класса, обеспечивающего компилирование, линкование и
        анализ шейдерных программ;
    \end{enumerate}
    \item \textbf{исходные} файлы:
    \begin{enumerate}[label=\arabic*), labelsep=0.5em]
        \item \text{shader\_extensions.cpp} -- реализация функций упрощения отправки данных в шейдерные программы;
        \item \text{shader.cpp} -- реализация класса, обеспечивающего компилирование, линкование и
        анализ шейдерных программ;
    \end{enumerate}
\end{itemize}

\section{Исходные файлы модуля \text{«ShadowMap»}}

Файлы модуля представлены ниже:

\begin{itemize}[label=---]
    \item \textbf{заголовочные} файлы:
    \begin{enumerate}[label=\arabic*), labelsep=0.5em]
        \item \text{DepthBufferGenerator.h} -- объявление функции создания теневой карты;
        \item \text{DepthBufferStruct.h} -- объевление структуры теневой карты;
        \item \text{ShadowMapMainRenderData.h} -- объявление структуры необходимых данных для
        стандартного алгоритма теневых карт;
        \item \text{ShadowMapPcfRenderData.h} -- объявление структуры необходимых данных для
        алгоритма теневых карт с фильтрацией (PCF);
        \item \text{ShadowMapNoiseRenderData.h} -- объявление структуры необходимых данных для
        алгоритма теневых карт с фильтрацией шумом (NOISE);
        \item \text{ShadowMapPcssRenderData.h} -- объявление структуры необходимых данных для
        алгоритма мягких теневых карт с фильтрацией (PCSS);
        \item \text{ShadowMapPcssNoiseRenderData.h} -- объявление структуры необходимых данных для
        алгоритма мягких теневых карт с фильтрацией шумом (PCSS-NOISE);
        \item \text{ShadowMap.h} -- объявление функций стандартного алгоритма теневых карт;
        \item \text{ShadowMapPcf.h} -- объявление функций алгоритма теневых карт с фильтрацией (PCF);
        \item \text{ShadowMapNoise.h} -- объявление функций алгоритма теневых карт с фильтрацией шумом (NOISE);
        \item \text{ShadowMapPcss.h} -- объявление функций алгоритма мягких теневых карт с фильтрацией (PCSS);
        \item \text{ShadowMapPcssNosie.h} -- объявление функций алгоритма мягких теневых карт с фильтрацией шумом (PCSS-NOISE);
    \end{enumerate}
    \item \textbf{исходные} файлы:
    \begin{enumerate}[label=\arabic*), labelsep=0.5em]
        \item \text{DepthBufferGenerator.cpp} -- реализация функции создания теневой карты;
        \item \text{ShadowMap.cpp} -- реализация функций стандартного алгоритма теневых карт;
        \item \text{ShadowMapPcf.cpp} -- реализация функций алгоритма теневых карт с фильтрацией (PCF);
        \item \text{ShadowMapNoise.cpp} -- реализация функций алгоритма теневых карт с фильтрацией шумом (NOISE);
        \item \text{ShadowMapPcss.cpp} -- реализация функций алгоритма мягких теневых карт с фильтрацией (PCSS);
        \item \text{ShadowMapPcssNosie.cpp} -- реализация функций алгоритма мягких теневых карт с фильтрацией шумом (PCSS-NOISE);
    \end{enumerate}
\end{itemize}

\section{Исходные файлы модуля \text{«app»}}

Файлы модуля представлены ниже:

\begin{itemize}[label=---]
    \item \textbf{заголовочные} файлы:
    \begin{enumerate}[label=\arabic*), labelsep=0.5em]
        \item \text{app\_args.h} -- объявление глобальных переменных приложения;
        \item \text{user\_msgs.h}
        \item \text{resource.h}
        \item \text{cpu\_stop\_watch.hpp};
        \item \text{cpu\_timing.h};
        \item \text{formater.h}
        \item \text{FpsSetterDialogProc.h}
        \item \text{general\_shadow\_options\_wnd\_proc.h}
        \item \text{Light.h}
        \item \text{lighting\_wnd\_proc.h}
        \item \text{LightStruct.h}
        \item \text{main\_wnd\_proc.h}
        \item \text{model\_wnd\_proc.h}
        \item \text{ModelLoader.h}
        \item \text{ModelLoaderDialogProc.h}
        \item \text{PlaneMeshGenerator.h}
        \item \text{ProjectionEnum.h}
        \item \text{render\_wnd\_proc.h}
        \item \text{ResolutionMapLimit.h}
        \item \text{shadow\_wnd\_proc.h}
        \item \text{ShadowAlgEnum.h}
        \item \text{shadowMap\_wnd\_proc.h}
        \item \text{time\_definers.h}
        \item \text{toolbar\_wnd\_proc.h}
    \end{enumerate}
    \item \textbf{исходные} файлы:
    \begin{enumerate}[label=\arabic*), labelsep=0.5em]
        \item \text{app\_args.cpp} -- объявление глобальных переменных приложения;
        \item \text{cpu\_timing.cpp};
        \item \text{formater.cpp}
        \item \text{FpsSetterDialogProc.cpp}
        \item \text{general\_shadow\_options\_wnd\_proc.cpp}
        \item \text{Light.cpp}
        \item \text{lighting\_wnd\_proc.cpp}
        \item \text{main\_wnd\_proc.cpp}
        \item \text{main.cpp}
        \item \text{model\_wnd\_proc.cpp}
        \item \text{ModelLoader.cpp}
        \item \text{ModelLoaderDialogProc.cpp}
        \item \text{PlaneMeshGenerator.cpp}
        \item \text{render\_wnd\_proc.cpp}
        \item \text{shadow\_wnd\_proc.cpp}
        \item \text{shadowMap\_wnd\_proc.cpp}
        \item \text{toolbar\_wnd\_proc.cpp}
    \end{enumerate}
\end{itemize}

\section*{Вывод}