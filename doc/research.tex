\chapter{Исследовательская часть}

В данном разделе будут представлены примеры работы программы,
проведены замеры процессорного времени и предоставлена информация
о технических характеристиках устройства.

\section{Примеры работы программы}

На рисунках \ref{img:example1} -- \ref{img:example2} представлен результат работы программы.

\begin{figure}[H]
	\centering
	\includegraphics[width=0.5\textwidth]{img/prog_1.jpg}
	\caption{Пример работы программы умножения матриц стандартным алгоритмом}
	\label{img:example1}
\end{figure}

\FloatBarrier

\begin{figure}[H]
	\centering
	\includegraphics[width=0.5\textwidth]{img/prog_2.jpg}
	\caption{Пример работы программы умножения матриц алгоритмом Копперсмита-Винограда}
	\label{img:example2}
\end{figure}

\FloatBarrier

\section{Время выполнения алгоритмов}

Для замера процессорного времени использовались функции:
\textit{QueryPerformanceFrequency} и \textit{QueryPerformanceCounter}
-- из библиотеки \textit{Windows.h}.
Возвращаемый результат -- время в наносекундах, число типа \textit{long long}.

Чтобы получить достаточно точное значение, производилось усреднение времени.
Количество запусков замера процессорного времени -- 5 замеров на одно умножение
для каждого алгоритма.

В замерах использовались квадратные матрицы с линейной размерностью от 25 до 1000,
с шагом 25. Для исследования худшего случая шаг также был равен 25, но в случае
четного итового рамера, берется размер меньший на единицу. Для лучшего случая
при нечетном размере, из итового вычитается единица.

В таблице \ref{tbl:entire_time} представлено процессорное время работы реализаций алгоритмов умножения матриц.

\begin{table}[h]
	\begin{center}
		\begin{threeparttable}
		\captionsetup{justification=raggedright, singlelinecheck=off}
		\caption{\centering\label{tbl:entire_time}Общий замер времени выполнения алгоритмов}
		\begin{tabular}{|r|r|r|r|}
			\hline
			& \multicolumn{3}{c|}{Время выполнения алгоритмов в м.с.} \\
			\hline
			Лин. размер квад. матриц & Стандартный & К.-В. & Опт. К.-В. \\
			\hline
			25 & 55 & 38 & 24 \\
			\hline
			50 & 433 & 330 & 194 \\
			\hline
			75 & 1495 & 952 & 638 \\
			\hline
			100 & 3544 & 2530 & 1501 \\
			\hline
			125 & 6906 & 4585 & 3188 \\
			\hline
			150 & 11984 & 7911 & 5402 \\
			\hline
			175 & 19259 & 12038 & 7637 \\
			\hline
			200 & 29878 & 17943 & 11481 \\
			\hline
			225 & 40972 & 26001 & 16517 \\
			\hline
			250 & 54911 & 35391 & 22438 \\
			\hline
			275 & 74568 & 46473 & 29652 \\
			\hline
			300 & 98190 & 61056 & 41697 \\
			\hline
			325 & 127718 & 76358 & 48715 \\
			\hline
			350 & 160399 & 96682 & 60879 \\
			\hline
			375 & 193909 & 117980 & 77352 \\
			\hline
			400 & 238757 & 142721 & 93249 \\
			\hline
			425 & 284285 & 174322 & 110690 \\
			\hline
			450 & 325922 & 202900 & 132289 \\
			\hline
			475 & 395771 & 241550 & 157438 \\
			\hline
			500 & 446783 & 280130 & 181781 \\
			\hline
			525 & 534853 & 329709 & 215798 \\
			\hline
			550 & 624734 & 373864 & 241986 \\
			\hline
			575 & 693049 & 430624 & 277352 \\
			\hline
			600 & 802146 & 482358 & 320674 \\
			\hline
			625 & 883541 & 554618 & 367441 \\
			\hline
			650 & 1024458 & 614565 & 410544 \\
			\hline
			675 & 1153367 & 693614 & 467557 \\
			\hline
			700 & 1277344 & 769613 & 504824 \\
			\hline
			725 & 1480026 & 866036 & 570906 \\
			\hline
			750 & 1567951 & 958241 & 639503 \\
			\hline
			775 & 1757306 & 1067806 & 710754 \\
			\hline
			800 & 1895453 & 1134108 & 756407 \\
			\hline
			825 & 2114161 & 1284740 & 853588 \\
			\hline
			850 & 2372030 & 1408853 & 918405 \\
			\hline
			875 & 2583694 & 1596526 & 1051980 \\
			\hline
			900 & 2775862 & 1688548 & 1093329 \\
			\hline
			925 & 3151542 & 1908621 & 1324578 \\
			\hline
			950 & 3458415 & 2116271 & 1459899 \\
			\hline
			975 & 3721142 & 2332657 & 1631293 \\
			\hline
			1000 & 3984835 & 2508923 & 1821546 \\
			\hline
		\end{tabular}
	\end{threeparttable}
\end{center}
\end{table}

\clearpage

В таблице \ref{tbl:bad_time} представлено процессорное время работы реализации алгоритмов
Копперсмита-Винограда и оптимизированного К.-В. для худшего случая.

\begin{table}[h]
	\begin{center}
		\begin{threeparttable}
		\captionsetup{justification=raggedright, singlelinecheck=off}
		\caption{\centering\label{tbl:bad_time}Замер времени выполнения алгоритма К.-В. и опт. К.-В. для худшего случая}
		\begin{tabular}{|r|r|r|}
		\hline
		& \multicolumn{2}{c|}{Время выполнения алгоритмов в м.с.} \\
		\hline
		Лин. размер квад. матриц & К.-В. & Опт. К.-В. \\
		\hline
		25 & 39 & 25 \\
		\hline
		49 & 303 & 170 \\
		\hline
		75 & 953 & 642 \\
		\hline
		99 & 2154 & 1630 \\
		\hline
		125 & 4386 & 3031 \\
		\hline
		149 & 7727 & 5057 \\
		\hline
		175 & 12404 & 7577 \\
		\hline
		199 & 17750 & 11235 \\
		\hline
		225 & 25440 & 16341 \\
		\hline
		249 & 34448 & 22422 \\
		\hline
		275 & 46464 & 29698 \\
		\hline
		299 & 59440 & 38776 \\
		\hline
		325 & 76765 & 50337 \\
		\hline
		349 & 94829 & 61567 \\
		\hline
		375 & 119310 & 80036 \\
		\hline
		399 & 140751 & 93474 \\
		\hline
		425 & 172217 & 110621 \\
		\hline
		449 & 203966 & 134952 \\
		\hline
		475 & 237212 & 159821 \\
		\hline
		499 & 275059 & 186279 \\
		\hline
		525 & 320534 & 216442 \\
		\hline
		549 & 374501 & 241869 \\
		\hline
		575 & 429901 & 286856 \\
		\hline
		599 & 488478 & 320890 \\
		\hline
		625 & 560422 & 363582 \\
		\hline
		649 & 629237 & 419438 \\
		\hline
		675 & 696557 & 461126 \\
		\hline
		699 & 791125 & 520665 \\
		\hline
		725 & 880298 & 583607 \\
		\hline
		749 & 976075 & 646468 \\
		\hline
		775 & 1066358 & 727178 \\
		\hline
		799 & 1182915 & 785664 \\
		\hline
		825 & 1300308 & 871023 \\
		\hline
		849 & 1407211 & 956524 \\
		\hline
		875 & 1578948 & 1041651 \\
		\hline
		899 & 1680469 & 1147553 \\
		\hline
		925 & 1858094 & 1260899 \\
		\hline
		949 & 1984525 & 1349930 \\
		\hline
		975 & 2243427 & 1576737 \\
		\hline
		999 & 2585201 & 1857755 \\
		\hline
		\end{tabular}
	\end{threeparttable}
\end{center}
\end{table}

\clearpage

В таблице \ref{tbl:best_time} представлено процессорное время работы реализации алгоритмов
Копперсмита-Винограда и оптимизированного К.-В. для лучшего случая.

\begin{table}[h]
	\begin{center}
		\begin{threeparttable}
		\captionsetup{justification=raggedright, singlelinecheck=off}
		\caption{\centering\label{tbl:best_time}Замер времени выполнения алгоритма К.-В. и опт. К.-В. для лучшего случая}
		\begin{tabular}{|r|r|r|}
		\hline
		& \multicolumn{2}{r|}{Время выполнения алгоритмов в м.с.} \\
		\hline
		Лин. размер квад. матриц & К.-В. & Опт. К.-В. \\
		\hline
		24 & 76 & 22 \\
		\hline
		50 & 286 & 195 \\
		\hline
		74 & 931 & 734 \\
		\hline
		100 & 2228 & 1485 \\
		\hline
		124 & 4256 & 2806 \\
		\hline
		150 & 7881 & 5020 \\
		\hline
		174 & 11892 & 7578 \\
		\hline
		200 & 18070 & 11904 \\
		\hline
		224 & 25051 & 16058 \\
		\hline
		250 & 34615 & 21975 \\
		\hline
		274 & 45884 & 28947 \\
		\hline
		300 & 59700 & 38207 \\
		\hline
		324 & 74821 & 48501 \\
		\hline
		350 & 95286 & 61931 \\
		\hline
		374 & 115465 & 73801 \\
		\hline
		400 & 141477 & 90674 \\
		\hline
		424 & 167673 & 111279 \\
		\hline
		450 & 202742 & 135502 \\
		\hline
		474 & 238482 & 157595 \\
		\hline
		500 & 279769 & 182335 \\
		\hline
		524 & 320138 & 205494 \\
		\hline
		550 & 375521 & 245692 \\
		\hline
		574 & 417601 & 280851 \\
		\hline
		600 & 479075 & 315510 \\
		\hline
		624 & 540319 & 358095 \\
		\hline
		650 & 615374 & 406110 \\
		\hline
		674 & 738236 & 465591 \\
		\hline
		700 & 852330 & 530819 \\
		\hline
		724 & 886080 & 652285 \\
		\hline
		750 & 1117122 & 724160 \\
		\hline
		774 & 1353057 & 771582 \\
		\hline
		800 & 1242630 & 812285 \\
		\hline
		824 & 1569935 & 1065781 \\
		\hline
		850 & 1995504 & 1133540 \\
		\hline
		874 & 1741485 & 1046188 \\
		\hline
		900 & 1699970 & 1325725 \\
		\hline
		924 & 2344866 & 1351606 \\
		\hline
		950 & 2470901 & 1548530 \\
		\hline
		974 & 3145604 & 2172560 \\
		\hline
		1000 & 2745771 & 1901390 \\
		\hline
		\end{tabular}
	\end{threeparttable}
\end{center}
\end{table}

\clearpage

На рисунках \ref{img:graph1} -- \ref{img:graph5} приведены результаты замеров процессорного времени.

\begin{figure}[ht]
	\centering
	\includegraphics[width=1.0\textwidth]{img/graph1.png}
	\caption{Сравнение всех алгоритмов умножения матриц}
	\label{img:graph1}
\end{figure}

\FloatBarrier

\begin{figure}[ht]
	\centering
	\includegraphics[width=1.0\textwidth]{img/graph2.png}
	\caption{Сравнение лучшего и худшего случаев для алгоритма Копперсмита-Винограда}
	\label{img:graph2}
\end{figure}

\FloatBarrier

\begin{figure}[ht]
	\centering
	\includegraphics[width=1.0\textwidth]{img/graph3.png}
	\caption{Сравнение лучшего и худшего случаев для оптимизированного алгоритма Копперсмита-Винограда}
	\label{img:graph3}
\end{figure}

\FloatBarrier

\begin{figure}[ht]
	\centering
	\includegraphics[width=1.0\textwidth]{img/graph4.png}
	\caption{Сравнение алгоритмов: К.-В. и оптимизированного К.-В. -- для худшего случая}
	\label{img:graph4}
\end{figure}

\FloatBarrier

\begin{figure}[ht]
	\centering
	\includegraphics[width=1.0\textwidth]{img/graph5.png}
	\caption{Сравнение алгоритмов: К.-В. и оптимизированного К.-В. -- для лучшего случая}
	\label{img:graph5}
\end{figure}

\FloatBarrier

\section{Технические характеристики устройства}

Ниже представлены характеристики устройства,
на котором проводилось тестирование программы:

\begin{itemize}
	\item Имя ОС -- Майкрософт Windows 10 Pro;
	\item Версия -- 10.0.19045 Сборка 19045
	\item Процессор	AMD Ryzen 7 4800H with Radeon Graphics, 2900 МГц, ядер: 8, логических процессоров: 16;
	\item RAM -- 16 Гб, 3200 МГц.
\end{itemize}

\section{Вывод}

В результате замеров процессорного времени выделены следущие аспекты:

\begin{itemize}
	\item Алгоритм Копперсмита-Винограда выполняется значительно быстрее (в 2 раза), засчет дополнительной памяти и
	уменьшения числа арифметических операций, но только для матриц с большими размерностями. Для матриц,
	ипользующиеся в компьтерной графике, выгоднее и эффективнее использовать аппаратное ускорение и параллельные вычисления,
	для нейронных сетей в свою очередь, размерность матриц никчемно мала и ровно так же быстрее будет использование
	аппаратных решений, например, аналоговые чипы;
	\item Оптимизированный Алгоритм Копперсмита-Винограда в общем случае быстрее в 1.5 раза обычного алгоритма Копперсмита-Винограда
	и в 3 раза стандартного алгоритма умножения матриц.
\end{itemize}

