\chapter{Конструкторский раздел}
\def \globalscale {1.0}

В данном разделе представлены схемы алгоритмов.

\section{Схемы алгоритмов модели освещения}

Схемы алгоритмов модели освещения представлены ниже.

\begin{enumerate}[label=\arabic*), labelsep=0.5em]
    \item Схема алгоритма Ламбертового освещения для точечного источника света (рисунок~\ref{chart:lambert_point}).
    \item Схема алгоритма Ламбертового освещения для прожекторного источника света (рисунок~\ref{chart:lambert_spot}).
\end{enumerate}

\begin{figure}
\centering
\begin{tikzpicture}[y=1cm, x=1cm, yscale=\globalscale,xscale=\globalscale, every node/.append style={scale=\globalscale}, inner sep=0pt, outer sep=0pt]
  \path[draw=black,miter limit=10.0,dash pattern=on 0.0794cm off 0.0794cm] (3.175, 15.1077) -- (3.7042, 15.1077);



  \path[draw=black,miter limit=10.0] (1.5875, 14.314) -- (1.5875, 13.7848);



  \path[draw=black,fill=white,rounded corners=0.8cm] (0.0, 15.9015) rectangle (3.175, 14.314);



  \begin{scope}[shift={(-0.0132, 0.0132)}]
    \node[text=black,anchor=south,fit={(0,0) (3.175, 1.5875)}] (text1662) at (1.5875, 15.0019){Начало};



  \end{scope}
  \path (3.7042, 16.4306) rectangle (4.736, 13.7848);



  \path[draw=black,miter limit=10.0] (4.736, 16.4306) -- (3.7042, 16.4306) -- (3.7042, 13.7848) -- (4.736, 13.7848);



  \begin{scope}[shift={(-0.0132, 0.0132)}]
    \node[text=black,anchor=south,fit={(0,0) (3.175, 1.5875)} ] (text6565) at (5.5, 15.0019){Алгоритм Ламбертового освещения для точечных источников света};



  \end{scope}
  \path[draw=black,miter limit=10.0] (1.5875, 12.4619) -- (1.5875, 11.9327);



  \path[draw=black,fill=white,miter limit=10.0] (0.5292, 13.7848) -- (2.6458, 13.7848) -- (3.175, 13.3615) -- (3.175, 12.4619) -- (0.0, 12.4619) -- (0.0, 13.3615) -- cycle;



  \begin{scope}[shift={(-0.0132, 0.0132)}]
    \node[text=black,anchor=south,fit={(0,0) (3.175, 1.5875)}] (text7727) at (1.5875, 13.0175){Обработка N пикселей};



  \end{scope}
  \path[draw=black,miter limit=10.0] (1.5875, 2.1677) -- (1.5875, 1.6386);



  \path[draw=black,fill=white,miter limit=10.0,cm={ -1.0,-0.0,0.0,-1.0,(3.175, 5.6584)}] (0.5292, 3.4906) -- (2.6458, 3.4906) -- (3.175, 3.0673) -- (3.175, 2.1677) -- (0.0, 2.1677) -- (0.0, 3.0673) -- cycle;



  \begin{scope}[shift={(-0.0132, 0.0132)}]
    \node[text=black,anchor=south,fit={(0,0) (3.175, 1.5875)}] (text9988) at (1.5875, 2.7252){Пока остались необработанные пиксели};



  \end{scope}
  \path[draw=black,miter limit=10.0] (1.5875, 10.3452) -- (1.5875, 9.816);



  \path[draw=black,miter limit=10.0,dash pattern=on 0.0794cm off 0.0794cm] (3.175, 11.139) -- (3.7042, 11.139);



  \path[draw=black,fill=white] (0.0, 11.9327) rectangle (3.175, 10.3452);



  \begin{scope}[shift={(-0.0132, 0.0132)}]
    \node[text=black,anchor=south,fit={(0,0) (3.175, 1.5875)}] (text7296) at (1.5875, 11.0331){$\vec{L}=norm(\vec{P_l}-\vec{P_f})$};



  \end{scope}
  \path[draw=black,miter limit=10.0,dash pattern=on 0.0794cm off 0.0794cm] (3.175, 9.0091) -- (3.7042, 9.0091);



  \path[draw=black,miter limit=10.0] (1.5875, 8.2153) -- (1.5875, 7.6994);



  \path[draw=black,fill=white] (0.0, 9.8028) rectangle (3.175, 8.2153);



  \begin{scope}[shift={(-0.0132, 0.0132)}]
    \node[text=black,anchor=south,fit={(0,0) (3.175, 1.5875)}] (text6853) at (1.5875, 8.9165){$I_{\text{diff}}=\max\begin{cases}\vec{L} \cdot \vec{N} \\ 0\end{cases}$};



  \end{scope}
  \path[draw=black,fill=white,rounded corners=0.8cm] (0.0, 1.6386) rectangle (3.175, 0.0511);



  \begin{scope}[shift={(-0.0132, 0.0132)}]
    \node[text=black,anchor=south,fit={(0,0) (3.175, 1.5875)}] (text6673) at (1.5875, 0.7408){Конец};



  \end{scope}
  \path (3.7042, 11.9081) rectangle (4.736, 10.3698);



  \path[draw=black,miter limit=10.0] (4.736, 11.9081) -- (3.7042, 11.9081) -- (3.7042, 10.3698) -- (4.736, 10.3698);



  \begin{scope}[shift={(-0.0132, 0.0132)}]
    \node[text=black,anchor=south,fit={(0,0) (3.175, 1.5875)} ] (text6100) at (5.5, 11.0331){Вычисление направления от пикселя к источнику света};



  \end{scope}
  \path (3.7042, 9.7782) rectangle (4.736, 8.2399);



  \path[draw=black,miter limit=10.0] (4.736, 9.7782) -- (3.7042, 9.7782) -- (3.7042, 8.2399) -- (4.736, 8.2399);



  \begin{scope}[shift={(-0.0132, 0.0132)}]
    \node[text=black,anchor=south,fit={(0,0) (3.175, 1.5875)} ] (text9922) at (5.5, 8.9165){Вычисление косинуса угла между нормалью и вектором $\vec{L}$};



  \end{scope}
  \path[draw=black,miter limit=10.0,dash pattern=on 0.0794cm off 0.0794cm] (3.175, 6.8908) -- (3.7042, 6.8858);



  \path[draw=black,miter limit=10.0] (1.5875, 6.1119) -- (1.5875, 5.6073);



  \path[draw=black,fill=white] (0.0, 7.6994) rectangle (3.175, 6.1119);



  \begin{scope}[shift={(-0.0132, 0.0132)}]
    \node[text=black,anchor=south,fit={(0,0) (3.175, 1.5875)}] (text8692) at (1.5875, 6.7998){$A=\max\begin{cases}1 - \frac{I_{\text{diff}}}{R} \\0\end{cases}$};



  \end{scope}
  \path (3.7042, 7.6502) rectangle (4.736, 6.1119);



  \path[draw=black,miter limit=10.0] (4.736, 7.6502) -- (3.7042, 7.6502) -- (3.7042, 6.1119) -- (4.736, 6.1119);



  \begin{scope}[shift={(-0.0132, 0.0132)}]
    \node[text=black,anchor=south,fit={(0,0) (3.175, 1.5875)} ] (text9266) at (5, 6.7733){Уменьшение интенсивности с расстоянием};



  \end{scope}
  \path[draw=black,miter limit=10.0] (1.5875, 4.0198) -- (1.5875, 3.4906);



  \path[draw=black,miter limit=10.0,dash pattern=on 0.0794cm off 0.0794cm] (3.175, 4.7987) -- (3.7042, 4.7937);



  \path[draw=black,fill=white] (0.0, 5.6073) rectangle (3.175, 4.0198);



  \begin{scope}[shift={(-0.0132, 0.0132)}]
    \node[text=black,anchor=south,fit={(0,0) (3.175, 1.5875)}] (text8086) at (1.5875, 4.7096){$\text{LightColor} = \text{Color} \cdot \text{Intensity} \cdot A \cdot I_{\text{diff}}$};



  \end{scope}
  \path (3.7042, 5.5581) rectangle (4.736, 4.0198);



  \path[draw=black,miter limit=10.0] (4.736, 5.5581) -- (3.7042, 5.5581) -- (3.7042, 4.0198) -- (4.736, 4.0198);



  \begin{scope}[shift={(-0.0132, 0.0132)}]
    \node[text=black,anchor=south,fit={(0,0) (3.175, 1.5875)},scale=0.8] (text1543) at (5.5, 4.6831){Результирующий цвет, где Intensity -- интенсивность источника света, Color -- цвет света};



  \end{scope}

\end{tikzpicture}

\caption{Схема алгоритма Ламбертового освещения для точечного источника света}
\label{chart:lambert_point}

\end{figure}

\FloatBarrier
\begin{figure}
\centering
\begin{tikzpicture}[y=1cm, x=1cm, yscale=\globalscale,xscale=\globalscale, every node/.append style={scale=\globalscale}, inner sep=0pt, outer sep=0pt]
  \path[draw=black,miter limit=10.0,dash pattern=on 0.0794cm off 0.0794cm] (3.175, 19.4469) -- (3.7042, 19.4469);



  \path[draw=black,miter limit=10.0] (1.5875, 18.6531) -- (1.5875, 18.124);



  \path[draw=black,fill=white,rounded corners=0.4445cm] (0.0, 20.2406) rectangle (3.175, 18.6531);



  \begin{scope}[shift={(-0.0132, 0.0132)}]
    \node[text=black,anchor=south,fit={(0,0) (3.175, 1.5875)}] (text7994) at (1.5875, 19.341){Начало};



  \end{scope}
  \path (3.7042, 20.7698) rectangle (4.736, 18.124);



  \path[draw=black,miter limit=10.0] (4.736, 20.7698) -- (3.7042, 20.7698) -- (3.7042, 18.124) -- (4.736, 18.124);



  \begin{scope}[shift={(-0.0132, 0.0132)}]
    \node[text=black,anchor=south,fit={(0,0) (3.175, 1.5875)} ] (text3136) at (5.5, 19.341){Алгоритм Ламбертового освещения для прожекторных источников света};



  \end{scope}
  \path[draw=black,miter limit=10.0] (1.5875, 16.801) -- (1.5875, 16.2719);



  \path[draw=black,fill=white,miter limit=10.0] (0.5292, 18.124) -- (2.6458, 18.124) -- (3.175, 17.7006) -- (3.175, 16.801) -- (0.0, 16.801) -- (0.0, 17.7006) -- cycle;



  \begin{scope}[shift={(-0.0132, 0.0132)}]
    \node[text=black,anchor=south,fit={(0,0) (3.175, 1.5875)}] (text5489) at (1.5875, 17.3567){Обработка N пикселей};



  \end{scope}
  \path[draw=black,miter limit=10.0] (1.5875, 2.1431) -- (1.5875, 1.614);



  \path[draw=black,fill=white,miter limit=10.0,cm={ -1.0,-0.0,0.0,-1.0,(3.175, 5.6092)}] (0.5292, 3.466) -- (2.6458, 3.466) -- (3.175, 3.0427) -- (3.175, 2.1431) -- (0.0, 2.1431) -- (0.0, 3.0427) -- cycle;



  \begin{scope}[shift={(-0.0132, 0.0132)}]
    \node[text=black,anchor=south,fit={(0,0) (3.175, 1.5875)}] (text8859) at (1.5875, 2.6988){Пока остались необработанные пиксели};



  \end{scope}
  \path[draw=black,miter limit=10.0] (1.5875, 14.6844) -- (1.5875, 14.1552);



  \path[draw=black,miter limit=10.0,dash pattern=on 0.0794cm off 0.0794cm] (3.175, 15.4781) -- (3.7042, 15.4781);



  \path[draw=black,fill=white] (0.0, 16.2719) rectangle (3.175, 14.6844);



  \begin{scope}[shift={(-0.0132, 0.0132)}]
    \node[text=black,anchor=south,fit={(0,0) (3.175, 1.5875)}] (text5326) at (1.5875, 15.3723){$\vec{L}=norm(\vec{P_l}-\vec{P_f})$};



  \end{scope}
  \path[draw=black,miter limit=10.0,dash pattern=on 0.0794cm off 0.0794cm] (3.175, 13.3482) -- (3.7042, 13.3482);



  \path[draw=black,miter limit=10.0] (1.5875, 12.5545) -- (1.5875, 12.0385);



  \path[draw=black,fill=white] (0.0, 14.142) rectangle (3.175, 12.5545);



  \begin{scope}[shift={(-0.0132, 0.0132)}]
    \node[text=black,anchor=south,fit={(0,0) (3.175, 1.5875)}] (text4645) at (1.5875, 13.2556){$I_{\text{diff}}=\max\begin{cases}\vec{L} \cdot \vec{N} \\ 0\end{cases}$};



  \end{scope}
  \path[draw=black,fill=white,rounded corners=0.4445cm] (0.0, 1.614) rectangle (3.175, 0.0265);



  \begin{scope}[shift={(-0.0132, 0.0132)}]
    \node[text=black,anchor=south,fit={(0,0) (3.175, 1.5875)}] (text1473) at (1.5875, 0.7144){Конец};



  \end{scope}
  \path (3.7042, 16.2473) rectangle (4.736, 14.709);



  \path[draw=black,miter limit=10.0] (4.736, 16.2473) -- (3.7042, 16.2473) -- (3.7042, 14.709) -- (4.736, 14.709);



  \begin{scope}[shift={(-0.0132, 0.0132)}]
    \node[text=black,anchor=south,fit={(0,0) (3.175, 1.5875)} ] (text1261) at (5.5, 15.3723){Вычисление
    направления от
    пикселя к источнику
    света};



  \end{scope}
  \path (3.7042, 14.1174) rectangle (4.736, 12.5791);



  \path[draw=black,miter limit=10.0] (4.736, 14.1174) -- (3.7042, 14.1174) -- (3.7042, 12.5791) -- (4.736, 12.5791);



  \begin{scope}[shift={(-0.0132, 0.0132)}]
    \node[text=black,anchor=south,fit={(0,0) (3.175, 1.5875)} ] (text5090) at (5.5, 13.2556){Вычисление косинуса угла между нормалью и вектором $\vec{L}$};



  \end{scope}
  \path[draw=black,miter limit=10.0,dash pattern=on 0.0794cm off 0.0794cm] (3.175, 11.23) -- (3.7042, 11.2249);



  \path[draw=black,miter limit=10.0] (1.5875, 10.451) -- (1.5875, 9.816);



  \path[draw=black,fill=white] (0.0, 12.0385) rectangle (3.175, 10.451);



  \begin{scope}[shift={(-0.0132, 0.0132)}]
    \node[text=black,anchor=south,fit={(0,0) (3.175, 1.5875)}] (text656) at (1.5875, 11.139){$A=\max\begin{cases}1 - \frac{I_{\text{diff}}}{R} \\0\end{cases}$};



  \end{scope}
  \path (3.7042, 11.9893) rectangle (4.736, 10.451);



  \path[draw=black,miter limit=10.0] (4.736, 11.9893) -- (3.7042, 11.9893) -- (3.7042, 10.451) -- (4.736, 10.451);



  \begin{scope}[shift={(-0.0132, 0.0132)}]
    \node[text=black,anchor=south,fit={(0,0) (3.175, 1.5875)} ] (text2294) at (5, 11.1125){Уменьшение интенсивности с расстоянием};



  \end{scope}
  \path[draw=black,miter limit=10.0] (1.5875, 3.9952) -- (1.5875, 3.466);



  \path[draw=black,miter limit=10.0,dash pattern=on 0.0794cm off 0.0794cm] (3.175, 4.7741) -- (3.7042, 4.7691);



  \path[draw=black,fill=white] (0.0, 5.5827) rectangle (3.175, 3.9952);



  \begin{scope}[shift={(-0.0132, 0.0132)}]
    \node[text=black,anchor=south,fit={(0,0) (3.175, 1.5875)}] (text2834) at (1.5875, 4.6831){$\text{LightColor} = \text{Color} \cdot \text{Intensity} \cdot \text{spotIntensity} \cdot A \cdot I_{\text{diff}}$};



  \end{scope}
  \path (3.7042, 5.5335) rectangle (4.736, 3.9952);



  \path[draw=black,miter limit=10.0] (4.736, 5.5335) -- (3.7042, 5.5335) -- (3.7042, 3.9952) -- (4.736, 3.9952);



  \begin{scope}[shift={(-0.0132, 0.0132)}]
    \node[text=black,anchor=south,fit={(0,0) (3.175, 1.5875)},scale=0.8] (text3823) at (5.5, 4.6567){Результирующий цвет, где Intensity -- интенсивность источника света, Color -- цвет света};



  \end{scope}
  \path[draw=black,miter limit=10.0] (1.5875, 8.2285) -- (1.5875, 7.6994);



  \path[draw=black,miter limit=10.0,dash pattern=on 0.0794cm off 0.0794cm] (3.175, 9.0223) -- (3.7042, 9.0223);



  \path[draw=black,fill=white] (0.0, 9.816) rectangle (3.175, 8.2285);



  \begin{scope}[shift={(-0.0132, 0.0132)}]
    \node[text=black,anchor=south,fit={(0,0) (3.175, 1.5875)}] (text8314) at (1.5875, 8.9165){$\text{spotEffect} = \vec{D} \cdot (-\vec{L})$};



  \end{scope}
  \path[draw=black,miter limit=10.0] (1.5875, 6.1119) -- (1.5875, 5.5827);



  \path[draw=black,miter limit=10.0,dash pattern=on 0.0794cm off 0.0794cm] (3.175, 6.9056) -- (3.7042, 6.9056);



  \path[draw=black,fill=white] (0.0, 7.6994) rectangle (3.175, 6.1119);



  \begin{scope}[shift={(-0.0132, 0.0132)}]
    \node[text=black,anchor=south,fit={(0,0) (3.175, 1.5875)},scale=0.55] (text8183) at (1.5875, 6.7998)
    {
      $
      \text{spotIntensity} = \min
      \begin{cases}
          \max
          \begin{cases}
              \frac{\text{spotEffect} - \cos(\theta_{\text{outer}})}{\cos(\theta_{\text{inner}}) - \cos(\theta_{\text{outer}})} \\
              0
          \end{cases} \\
          1
      \end{cases}
      $
    };



  \end{scope}
  \path (3.7042, 9.7914) rectangle (4.6567, 8.2531);



  \path[draw=black,miter limit=10.0] (4.6567, 9.7914) -- (3.7042, 9.7914) -- (3.7042, 8.2531) -- (4.6567, 8.2531);



  \begin{scope}[shift={(-0.0132, 0.0132)}]
    \node[text=black,anchor=south,fit={(0,0) (3.175, 1.5875)} ] (text6054) at (5.5, 8.9165){Формирование направления интенсивности света прожектора};



  \end{scope}
  \path (3.7042, 7.6748) rectangle (4.6567, 6.1365);



  \path[draw=black,miter limit=10.0] (4.6567, 7.6748) -- (3.7042, 7.6748) -- (3.7042, 6.1365) -- (4.6567, 6.1365);



  \begin{scope}[shift={(-0.0132, 0.0132)}]
    \node[text=black,anchor=south,fit={(0,0) (3.175, 1.5875)} ] (text8697) at (5.5, 6.7998){Смягчение рассеивания обода прожекторного луча};



  \end{scope}

\end{tikzpicture}

\caption{Схема алгоритма Ламбертового освещения для прожекторного источника света}
\label{chart:lambert_spot}

\end{figure}

\FloatBarrier

\section{Схемы алгоритмов теневых карт}

Схемы алгоритмов теневых карт представлены ниже.

\begin{enumerate}[label=\arabic*), labelsep=0.5em]
    \item Схема алгоритма теневых карт (\hbox{рисунок~\ref{chart:shadow_map}}).
    \item Схема алгоритма теневых карт с фильтрацией (PCF) (\hbox{рисунок~\ref{chart:shadow_map_pcf}}).
    \item Схема алгоритма теневых карт с фильтрацией шумом (NOISE) (\hbox{рисунок~\ref{chart:shadow_map_noise}}).
    \item Схема алгоритма мягких теневых карт с фильтрацией (PCSS) (\hbox{рисунок~\ref{chart:shadow_map_pcss}}).
    \item Схема алгоритма мягких теневых карт с фильтрацией шумом \hbox{(PCSS-NOISE)} (\hbox{рисунок~\ref{chart:shadow_map_pcss_noise}}).
\end{enumerate}

\begin{figure}
\centering
\begin{tikzpicture}[y=1cm, x=1cm, yscale=\globalscale*0.8, every node/.style={font=\scriptsize},xscale=\globalscale*0.8, every node/.style={font=\scriptsize}, every node/.append style={scale=\globalscale}, inner sep=0pt, outer sep=0pt]
  \path[draw=black,miter limit=10.0,dash pattern=on 0.0794cm off 0.0794cm] (16.4042, 16.9558) -- (16.6688, 16.9558) -- (16.9333, 16.955);



  \path[draw=black,fill=white,dash pattern=on 0.0794cm off 0.0794cm] (12.1708, 19.0651) rectangle (16.4042, 14.845);



  \path[draw=black,miter limit=10.0,dash pattern=on 0.0794cm off 0.0794cm] (4.2333, 11.67) -- (4.7625, 11.6567);



  \path[draw=black,fill=white,dash pattern=on 0.0794cm off 0.0794cm] (0.0, 16.9616) rectangle (4.2333, 6.3783);



  \path[draw=black,miter limit=10.0,dash pattern=on 0.0794cm off 0.0794cm] (3.7042, 17.9935) -- (4.7625, 17.9935);



  \path[draw=black,miter limit=10.0] (2.1167, 17.1998) -- (2.1167, 16.6706);



  \path[draw=black,fill=white,rounded corners=0.4445cm] (0.5292, 18.7873) rectangle (3.7042, 17.1998);



  \begin{scope}[shift={(-0.0132, 0.0132)}]
    \node[text=black,anchor=south,fit={(0,0) (3.175, 1.5875)}] (text2549) at (2.1167, 17.8858){Начало};



  \end{scope}
  \path (4.7625, 19.3164) rectangle (5.7944, 16.6706);



  \path[draw=black,miter limit=10.0] (5.7944, 19.3164) -- (4.7625, 19.3164) -- (4.7625, 16.6706) -- (5.7944, 16.6706);



  \begin{scope}[shift={(-0.0132, 0.0132)}]
    \node[text=black,anchor=south,fit={(0,0) (3.175, 1.5875)}] (text2469) at (6, 17.8858){Алгоритм стандартных теневых карт};



  \end{scope}
  \path[draw=black,miter limit=10.0] (2.1167, 15.0831) -- (2.1167, 14.5672);



  \path[draw=black,fill=white] (0.5292, 16.6706) rectangle (3.7042, 15.0831);



  \begin{scope}[shift={(-0.0132, 0.0132)}]
    \node[text=black,anchor=south,fit={(0,0) (3.175, 1.5875)}] (text2966) at (2.1167, 15.7692){Инициализировать теневую карту};



  \end{scope}
  \path[draw=black,miter limit=10.0] (2.1167, 12.9797) -- (2.1167, 12.4637);



  \path[draw=black,fill=white] (0.5292, 14.5672) rectangle (3.7042, 12.9797);



  \begin{scope}[shift={(-0.0132, 0.0132)}]
    \node[text=black,anchor=south,fit={(0,0) (3.175, 1.5875)}] (text6546) at (2.1167, 13.679){Инициализировать источник света};



  \end{scope}
  \path[draw=black,miter limit=10.0] (2.1167, 10.8762) -- (2.1167, 10.3471);



  \path[draw=black,fill=white] (0.5292, 12.4637) rectangle (3.7042, 10.8762);



  \begin{scope}[shift={(-0.0132, 0.0132)}]
    \node[text=black,anchor=south,fit={(0,0) (3.175, 1.5875)}] (text1082) at (2.1167, 11.5623){Инициализировать камеру};



  \end{scope}
  \path[draw=black,miter limit=10.0] (2.1167, 8.7596) -- (2.1167, 8.2304);



  \path[draw=black,fill=white] (0.5292, 10.3471) rectangle (3.7042, 8.7596);



  \begin{scope}[shift={(-0.0132, 0.0132)}]
    \node[text=black,anchor=south,fit={(0,0) (3.175, 1.5875)}] (text8917) at (2.1167, 9.4456){Инициализировать модели};



  \end{scope}
  \path (4.7625, 13.2064) rectangle (5.7944, 10.1071);



  \path[draw=black,miter limit=10.0] (5.7944, 13.2064) -- (4.7625, 13.2064) -- (4.7625, 10.1071) -- (5.7944, 10.1071);



  \begin{scope}[shift={(-0.0132, 0.0132)}]
    \node[text=black,anchor=south,fit={(0,0) (3.175, 1.5875)}] (text7521) at (6.5, 11.5623){Подготовительный этап, инициализирующий минимально необходимые данные};



  \end{scope}
  \path[draw=black,miter limit=10.0] (2.1167, 6.6429) -- (2.1167, 6.1132) -- (8.9958, 6.1132) -- (8.9958, 19.3424) -- (14.2875, 19.3424) -- (14.2875, 18.7873);



  \path[draw=black,fill=white] (0.5292, 8.2304) rectangle (3.7042, 6.6429);



  \begin{scope}[shift={(-0.0132, 0.0132)}]
    \node[text=black,anchor=south,fit={(0,0) (3.175, 1.5875)}] (text5316) at (2.1167, 7.329){Инициализировать
шейдеры};



  \end{scope}
  \path[draw=black,miter limit=10.0] (14.2875, 17.1998) -- (14.2875, 16.6706);



  \path[draw=black,fill=white] (12.7, 18.7873) rectangle (15.875, 17.1998);



  \begin{scope}[shift={(-0.0132, 0.0132)}]
    \node[text=black,anchor=south,fit={(0,0) (3.175, 1.5875)}] (text918) at (14.2875, 17.8858){Нарисовать сцену с точки зрения источника света};



  \end{scope}
  \path[draw=black,miter limit=10.0] (14.2875, 15.0831) -- (14.2875, 14.3158);



  \path[draw=black,fill=white] (12.7, 16.6706) rectangle (15.875, 15.0831);



  \begin{scope}[shift={(-0.0132, 0.0132)}]
    \node[text=black,anchor=south,fit={(0,0) (3.175, 1.5875)}] (text5470) at (14.2875, 15.7692){Заполнить теневую карту значениями глубины};



  \end{scope}
  \path (16.9333, 18.2779) rectangle (17.9652, 15.6321);



  \path[draw=black,miter limit=10.0] (17.9652, 18.2779) -- (16.9333, 18.2779) -- (16.9333, 15.6321) -- (17.9652, 15.6321);



  \begin{scope}[shift={(-0.0132, 0.0132)}]
    \node[text=black,anchor=south,fit={(0,0) (3.175, 1.5875)} ] (text9424) at (18.5, 16.854){Этап заполнения теневой карты};



  \end{scope}
  \path[draw=black,miter limit=10.0] (14.2875, 12.9929) -- (14.2875, 12.4637);



  \path[draw=black,fill=white,miter limit=10.0] (13.2292, 14.3158) -- (15.3458, 14.3158) -- (15.875, 13.8925) -- (15.875, 12.9929) -- (12.7, 12.9929) -- (12.7, 13.8925) -- cycle;



  \begin{scope}[shift={(-0.0132, 0.0132)}]
    \node[text=black,anchor=south,fit={(0,0) (3.175, 1.5875)}] (text8252) at (14.2875, 13.5467){Обработка N пикселей};



  \end{scope}
  \path[draw=black,miter limit=10.0] (14.2875, 2.145) -- (14.2875, 1.6158);



  \path[draw=black,fill=white,miter limit=10.0,cm={ -1.0,-0.0,0.0,-1.0,(28.575, 5.6129)}] (13.2292, 3.4679) -- (15.3458, 3.4679) -- (15.875, 3.0446) -- (15.875, 2.145) -- (12.7, 2.145) -- (12.7, 3.0446) -- cycle;



  \begin{scope}[shift={(-0.0132, 0.0132)}]
    \node[text=black,anchor=south,fit={(0,0) (3.175, 1.5875)}] (text9232) at (14.2875, 2.6988){Пока остались необработанные пиксели};



  \end{scope}
  \path[draw=black,miter limit=10.0] (14.2875, 10.8762) -- (14.2875, 10.3471);



  \path[draw=black,miter limit=10.0,dash pattern=on 0.0794cm off 0.0794cm] (15.875, 11.67) -- (16.6688, 11.67);



  \path[draw=black,fill=white] (12.7, 12.4637) rectangle (15.875, 10.8762);



  \begin{scope}[shift={(-0.0132, 0.0132)}]
    \node[text=black,anchor=south,fit={(0,0) (3.175, 1.5875)}] (text6919) at (14.2875, 11.5623){Преобразовать координаты фрагмента в пространство света};



  \end{scope}
  \path[draw=black,miter limit=10.0,dash pattern=on 0.0794cm off 0.0794cm] (15.875, 9.5401) -- (16.6688, 9.5401);



  \path[draw=black,miter limit=10.0] (14.2875, 8.7463) -- (14.2875, 8.4825) -- (14.2875, 8.2304);



  \path[draw=black,fill=white] (12.7, 10.3338) rectangle (15.875, 8.7463);



  \begin{scope}[shift={(-0.0132, 0.0132)}]
    \node[text=black,anchor=south,fit={(0,0) (3.175, 1.5875)}] (text2069) at (14.2875, 9.4456){Считать соответсвующее значение глубины из теневой карты};



  \end{scope}
  \path (16.6688, 12.477) rectangle (17.7271, 10.863);



  \path[draw=black,miter limit=10.0] (17.7271, 12.477) -- (16.6688, 12.477) -- (16.6688, 10.863) -- (17.7271, 10.863);



  \begin{scope}[shift={(-0.0132, 0.0132)}]
    \node[text=black,anchor=south,fit={(0,0) (3.175, 1.5875)} ] (text3324) at (17.5, 11.5623){Далее
d1};



  \end{scope}
  \path (16.6688, 10.3471) rectangle (17.7006, 8.7331);



  \path[draw=black,miter limit=10.0] (17.7006, 10.3471) -- (16.6688, 10.3471) -- (16.6688, 8.7331) -- (17.7006, 8.7331);



  \begin{scope}[shift={(-0.0132, 0.0132)}]
    \node[text=black,anchor=south,fit={(0,0) (3.175, 1.5875)} ] (text7715) at (17.5, 9.4456){Далее d2};



  \end{scope}
  \path[draw=black,miter limit=10.0] (15.875, 7.4366) -- (16.3777, 7.4374) -- (16.3777, 6.2823);



  \path[draw=black,fill=black,miter limit=10.0] (16.3777, 6.1434) -- (16.2851, 6.3286) -- (16.3777, 6.2823) -- (16.4703, 6.3286) -- cycle;



  \begin{scope}[shift={(-0.0132, 0.0132)}]
    \node[text=black,anchor=south,fit={(0,0) (3.175, 1.5875)}] (text5409) at (16.1396, 7.62){Да};



  \end{scope}
  \path[draw=black,miter limit=10.0] (12.7, 7.4366) -- (12.1708, 7.4374) -- (12.1708, 6.2823);



  \path[draw=black,fill=black,miter limit=10.0] (12.1708, 6.1434) -- (12.0782, 6.3286) -- (12.1708, 6.2823) -- (12.2634, 6.3286) -- cycle;



  \path[draw=black,fill=white,miter limit=10.0] (14.2875, 8.2304) -- (15.875, 7.4366) -- (14.2875, 6.6429) -- (12.7, 7.4366) -- cycle;



  \begin{scope}[shift={(-0.0132, 0.0132)}]
    \node[text=black,anchor=south,fit={(0,0) (3.175, 1.5875)}] (text6167) at (14.2875, 7.329){d1 > d2};



  \end{scope}
  \path[draw=black,miter limit=10.0] (16.3777, 4.5262) -- (16.3777, 3.9979) -- (14.2875, 3.9979) -- (14.2875, 3.6364);



  \path[draw=black,fill=black,miter limit=10.0] (14.2875, 3.4975) -- (14.1949, 3.6827) -- (14.2875, 3.6364) -- (14.3801, 3.6827) -- cycle;



  \path[draw=black,fill=white] (14.7902, 6.1137) rectangle (17.9652, 4.5262);



  \path[draw=black,miter limit=10.0] (15.1077, 6.1137) -- (15.1077, 4.5262)(17.6477, 6.1137) -- (17.6477, 4.5262);



  \begin{scope}[shift={(-0.0132, 0.0132)}]
    \node[text=black,anchor=south,fit={(0,0) (3.175, 1.5875)}] (text4721) at (16.3777, 5.2123){Применить\\фоновое\\освещение};



  \end{scope}
  \path[draw=black,miter limit=10.0] (12.1708, 4.5262) -- (12.1708, 3.9979) -- (14.2875, 3.9979) -- (14.2875, 3.6364);



  \path[draw=black,fill=black,miter limit=10.0] (14.2875, 3.4975) -- (14.1949, 3.6827) -- (14.2875, 3.6364) -- (14.3801, 3.6827) -- cycle;



  \path[draw=black,fill=white] (10.5833, 6.1137) rectangle (13.7583, 4.5262);



  \path[draw=black,miter limit=10.0] (10.9008, 6.1137) -- (10.9008, 4.5262)(13.4408, 6.1137) -- (13.4408, 4.5262);



  \begin{scope}[shift={(-0.0132, 0.0132)}]
    \node[text=black,anchor=south,fit={(0,0) (3.175, 1.5875)}] (text3281) at (12.1708, 5.2123){Применить\\полное\\освещение};



  \end{scope}
  \path[draw=black,fill=white,rounded corners=0.4445cm] (12.7, 1.6158) rectangle (15.875, 0.0283);



  \begin{scope}[shift={(-0.0132, 0.0132)}]
    \node[text=black,anchor=south,fit={(0,0) (3.175, 1.5875)}] (text6452) at (14.2875, 0.7144){Конец};



  \end{scope}

\end{tikzpicture}

\caption{Схема стандартного алгоритма теневых карт}
\label{chart:shadow_map}

\end{figure}

\FloatBarrier
\begin{figure}
\centering
\begin{tikzpicture}[y=1cm, x=1cm, yscale=\globalscale*0.8,xscale=\globalscale*0.8, every node/.style={font=\scriptsize}, every node/.append style={scale=\globalscale}, inner sep=0pt, outer sep=0pt]
  \path[draw=black,miter limit=10.0,dash pattern=on 0.0794cm off 0.0794cm] (16.4042, 21.2148) -- (16.6688, 21.2148) -- (16.9333, 21.2148);



  \path[draw=black,fill=white,dash pattern=on 0.0794cm off 0.0794cm] (12.1708, 23.3249) rectangle (16.4042, 19.1048);



  \path[draw=black,miter limit=10.0,dash pattern=on 0.0794cm off 0.0794cm] (4.2333, 15.9298) -- (4.7625, 15.9165);



  \path[draw=black,fill=white,dash pattern=on 0.0794cm off 0.0794cm] (0.0, 21.2214) rectangle (4.2333, 10.6381);



  \path[draw=black,miter limit=10.0,dash pattern=on 0.0794cm off 0.0794cm] (3.7042, 22.2533) -- (4.7625, 22.2533);



  \path[draw=black,miter limit=10.0] (2.1167, 21.4596) -- (2.1167, 20.9304);



  \path[draw=black,fill=white,rounded corners=0.4445cm] (0.5292, 23.0471) rectangle (3.7042, 21.4596);



  \begin{scope}[shift={(-0.0132, 0.0132)}]
    \node[text=black,anchor=south,fit={(0,0) (3.175, 1.5875)}] (text830) at (2.1167, 22.1456){Начало};



  \end{scope}
  \path (4.7625, 23.5762) rectangle (5.7944, 20.9304);



  \path[draw=black,miter limit=10.0] (5.7944, 23.5762) -- (4.7625, 23.5762) -- (4.7625, 20.9304) -- (5.7944, 20.9304);



  \begin{scope}[shift={(-0.0132, 0.0132)}]
    \node[text=black,anchor=south,fit={(0,0) (3.175, 1.5875)} ] (text9586) at (6.5, 22.1456){Алгоритм теневых карт с фильтрацией (PCF)};



  \end{scope}
  \path[draw=black,miter limit=10.0] (2.1167, 19.3429) -- (2.1167, 18.827);



  \path[draw=black,fill=white] (0.5292, 20.9304) rectangle (3.7042, 19.3429);



  \begin{scope}[shift={(-0.0132, 0.0132)}]
    \node[text=black,anchor=south,fit={(0,0) (3.175, 1.5875)}] (text741) at (2.1167, 20.029){Инициализировать теневую карту};



  \end{scope}
  \path[draw=black,miter limit=10.0] (2.1167, 17.2395) -- (2.1167, 16.7235);



  \path[draw=black,fill=white] (0.5292, 18.827) rectangle (3.7042, 17.2395);



  \begin{scope}[shift={(-0.0132, 0.0132)}]
    \node[text=black,anchor=south,fit={(0,0) (3.175, 1.5875)}] (text7418) at (2.1167, 17.9387){Инициализировать источник света};



  \end{scope}
  \path[draw=black,miter limit=10.0] (2.1167, 15.136) -- (2.1167, 14.6069);



  \path[draw=black,fill=white] (0.5292, 16.7235) rectangle (3.7042, 15.136);



  \begin{scope}[shift={(-0.0132, 0.0132)}]
    \node[text=black,anchor=south,fit={(0,0) (3.175, 1.5875)}] (text6602) at (2.1167, 15.8221){Инициализировать камеру};



  \end{scope}
  \path[draw=black,miter limit=10.0] (2.1167, 13.0194) -- (2.1167, 12.4902);



  \path[draw=black,fill=white] (0.5292, 14.6069) rectangle (3.7042, 13.0194);



  \begin{scope}[shift={(-0.0132, 0.0132)}]
    \node[text=black,anchor=south,fit={(0,0) (3.175, 1.5875)}] (text9819) at (2.1167, 13.7054){Инициализировать модели};



  \end{scope}
  \path (4.7625, 17.4662) rectangle (5.7944, 14.3669);



  \path[draw=black,miter limit=10.0] (5.7944, 17.4662) -- (4.7625, 17.4662) -- (4.7625, 14.3669) -- (5.7944, 14.3669);



  \begin{scope}[shift={(-0.0132, 0.0132)}]
    \node[text=black,anchor=south,fit={(0,0) (3.175, 1.5875)} ] (text7887) at (6.5, 15.8221){Подготовительный этап, инициализирующий минимально необходимые};



  \end{scope}
  \path[draw=black,miter limit=10.0] (2.1167, 10.9027) -- (2.1167, 10.3701) -- (8.6254, 10.3701) -- (8.6254, 23.5992) -- (14.2875, 23.5992) -- (14.2875, 23.0471);



  \path[draw=black,fill=white] (0.5292, 12.4902) rectangle (3.7042, 10.9027);



  \begin{scope}[shift={(-0.0132, 0.0132)}]
    \node[text=black,anchor=south,fit={(0,0) (3.175, 1.5875)}] (text6833) at (2.1167, 11.5887){Инициализировать
шейдеры};



  \end{scope}
  \path[draw=black,miter limit=10.0] (14.2875, 21.4596) -- (14.2875, 20.9304);



  \path[draw=black,fill=white] (12.7, 23.0471) rectangle (15.875, 21.4596);



  \begin{scope}[shift={(-0.0132, 0.0132)}]
    \node[text=black,anchor=south,fit={(0,0) (3.175, 1.5875)}] (text7755) at (14.2875, 22.1456){Нарисовать сцену с точки зрения источника света};



  \end{scope}
  \path[draw=black,miter limit=10.0] (14.2875, 19.3429) -- (14.2875, 18.5756);



  \path[draw=black,fill=white] (12.7, 20.9304) rectangle (15.875, 19.3429);



  \begin{scope}[shift={(-0.0132, 0.0132)}]
    \node[text=black,anchor=south,fit={(0,0) (3.175, 1.5875)}] (text3051) at (14.2875, 20.029){Заполнить теневую карту значениями глубины};



  \end{scope}
  \path (16.9333, 22.5377) rectangle (17.9652, 19.8919);



  \path[draw=black,miter limit=10.0] (17.9652, 22.5377) -- (16.9333, 22.5377) -- (16.9333, 19.8919) -- (17.9652, 19.8919);



  \begin{scope}[shift={(-0.0132, 0.0132)}]
    \node[text=black,anchor=south,fit={(0,0) (3.175, 1.5875)} ] (text4923) at (18.5, 21.1137){Этап заполнения теневой карты};



  \end{scope}
  \path[draw=black,miter limit=10.0] (14.2875, 17.2527) -- (14.2875, 16.7235);



  \path[draw=black,fill=white,miter limit=10.0] (13.2292, 18.5756) -- (15.3458, 18.5756) -- (15.875, 18.1523) -- (15.875, 17.2527) -- (12.7, 17.2527) -- (12.7, 18.1523) -- cycle;



  \begin{scope}[shift={(-0.0132, 0.0132)}]
    \node[text=black,anchor=south,fit={(0,0) (3.175, 1.5875)}] (text4800) at (14.2875, 17.8065){Обработка N пикселей};



  \end{scope}
  \path[draw=black,miter limit=10.0,dash pattern=on 0.0794cm off 0.0794cm] (15.875, 15.9298) -- (16.6688, 15.9298);



  \path[draw=black,miter limit=10.0] (14.2875, 15.136) -- (14.2875, 14.6069);



  \path[draw=black,fill=white] (12.7, 16.7235) rectangle (15.875, 15.136);



  \begin{scope}[shift={(-0.0132, 0.0132)}]
    \node[text=black,anchor=south,fit={(0,0) (3.175, 1.5875)}] (text1572) at (14.2875, 15.8221){Преобразовать координаты фрагмента в пространство света};



  \end{scope}
  \path (16.6688, 16.7367) rectangle (17.7271, 15.1228);



  \path[draw=black,miter limit=10.0] (17.7271, 16.7367) -- (16.6688, 16.7367) -- (16.6688, 15.1228) -- (17.7271, 15.1228);



  \begin{scope}[shift={(-0.0132, 0.0132)}]
    \node[text=black,anchor=south,fit={(0,0) (3.175, 1.5875)} ] (text2822) at (18, 15.8221){Далее
d1};



  \end{scope}
  \path[draw=black,fill=white,rounded corners=0.4445cm] (0.5292, 3.2544) rectangle (3.7042, 1.6669);



  \begin{scope}[shift={(-0.0132, 0.0132)}]
    \node[text=black,anchor=south,fit={(0,0) (3.175, 1.5875)}] (text75) at (2.1167, 2.3548){Конец};



  \end{scope}
  \path[draw=black,miter limit=10.0] (14.2875, 11.1673) -- (14.2875, 10.6381);



  \path[draw=black,fill=white,miter limit=10.0] (13.2292, 12.4902) -- (15.3458, 12.4902) -- (15.875, 12.0669) -- (15.875, 11.1673) -- (12.7, 11.1673) -- (12.7, 12.0669) -- cycle;



  \begin{scope}[shift={(-0.0132, 0.0132)}]
    \node[text=black,anchor=south,fit={(0,0) (3.175, 1.5875)},scale=0.9] (text4690) at (14.2875, 11.721){Обработка MxM окаймляющих пикселей};



  \end{scope}
  \path[draw=black,miter limit=10.0] (14.2875, 13.0194) -- (14.2875, 12.4902);



  \path[draw=black,miter limit=10.0,dash pattern=on 0.0794cm off 0.0794cm] (15.875, 13.806) -- (16.6688, 13.8023);



  \path[draw=black,fill=white] (12.7, 14.6069) rectangle (15.875, 13.0194);



  \begin{scope}[shift={(-0.0132, 0.0132)}]
    \node[text=black,anchor=south,fit={(0,0) (3.175, 1.5875)}] (text5632) at (14.2875, 13.7054){aver = 0};



  \end{scope}
  \path[draw=black,miter limit=10.0,dash pattern=on 0.0794cm off 0.0794cm] (15.875, 9.8515) -- (16.6688, 9.8552);



  \path[draw=black,miter limit=10.0] (14.2875, 9.0506) -- (14.2875, 8.5196);



  \path[draw=black,fill=white] (12.7, 10.6381) rectangle (15.875, 9.0506);



  \begin{scope}[shift={(-0.0132, 0.0132)}]
    \node[text=black,anchor=south,fit={(0,0) (3.175, 1.5875)}] (text8315) at (14.2875, 9.7367){Считать соответствующую глубину в окрестности};



  \end{scope}
  \path (16.6688, 10.6646) rectangle (17.7271, 9.0506);



  \path[draw=black,miter limit=10.0] (17.7271, 10.6646) -- (16.6688, 10.6646) -- (16.6688, 9.0506) -- (17.7271, 9.0506);



  \begin{scope}[shift={(-0.0132, 0.0132)}]
    \node[text=black,anchor=south,fit={(0,0) (3.175, 1.5875)} ] (text240) at (18, 9.7631){Далее
d2};



  \end{scope}
  \path (16.6688, 14.6069) rectangle (17.7271, 12.9929);



  \path[draw=black,miter limit=10.0] (17.7271, 14.6069) -- (16.6688, 14.6069) -- (16.6688, 12.9929) -- (17.7271, 12.9929);



  \begin{scope}[shift={(-0.0132, 0.0132)}]
    \node[text=black,anchor=south,fit={(0,0) (3.175, 1.5875)} ] (text3334) at (18.5, 13.7054){Количество перекрываемых глубин};



  \end{scope}
  \path[draw=black,miter limit=10.0] (15.875, 7.7258) -- (16.3761, 7.7242) -- (16.3761, 6.5979);



  \path[draw=black,fill=black,miter limit=10.0] (16.3761, 6.459) -- (16.2835, 6.6442) -- (16.3761, 6.5979) -- (16.4687, 6.6442) -- cycle;



  \begin{scope}[shift={(-0.0132, 0.0132)}]
    \node[text=black,anchor=south,fit={(0,0) (3.175, 1.5875)}] (text4706) at (16.0602, 7.911){Да};



  \end{scope}
  \path[draw=black,miter limit=10.0] (12.7, 7.7258) -- (12.065, 7.7242) -- (12.065, 4.3127) -- (14.2875, 4.3127) -- (14.2875, 3.9521);



  \path[draw=black,fill=black,miter limit=10.0] (14.2875, 3.8132) -- (14.1949, 3.9984) -- (14.2875, 3.9521) -- (14.3801, 3.9984) -- cycle;



  \path[draw=black,fill=white,miter limit=10.0] (14.2875, 8.5196) -- (15.875, 7.7258) -- (14.2875, 6.9321) -- (12.7, 7.7258) -- cycle;



  \begin{scope}[shift={(-0.0132, 0.0132)}]
    \node[text=black,anchor=south,fit={(0,0) (3.175, 1.5875)}] (text9537) at (14.2875, 7.62){d1 > d2};



  \end{scope}
  \path[draw=black,miter limit=10.0] (16.3761, 4.8419) -- (16.3761, 4.3127) -- (14.2875, 4.3127) -- (14.2875, 3.9521);



  \path[draw=black,fill=black,miter limit=10.0] (14.2875, 3.8132) -- (14.1949, 3.9984) -- (14.2875, 3.9521) -- (14.3801, 3.9984) -- cycle;



  \path[draw=black,fill=white] (14.7902, 6.4294) rectangle (17.9652, 4.8419);



  \begin{scope}[shift={(-0.0132, 0.0132)}]
    \node[text=black,anchor=south,fit={(0,0) (3.175, 1.5875)}] (text395) at (16.3777, 5.5298){aver += 1};



  \end{scope}
  \path[draw=black,miter limit=10.0] (14.2875, 2.4606) -- (14.2875, 1.9315);



  \path[draw=black,fill=white,miter limit=10.0,cm={ -1.0,-0.0,0.0,-1.0,(28.575, 6.2442)}] (13.2292, 3.7835) -- (15.3458, 3.7835) -- (15.875, 3.3602) -- (15.875, 2.4606) -- (12.7, 2.4606) -- (12.7, 3.3602) -- cycle;



  \begin{scope}[shift={(-0.0132, 0.0132)}]
    \node[text=black,anchor=south,fit={(0,0) (3.175, 1.5875)}] (text193) at (14.2875, 3.0162){Пока остались пиксели в окрестности};



  \end{scope}
  \path[draw=black,miter limit=10.0,dash pattern=on 0.0794cm off 0.0794cm] (3.7042, 7.7258) -- (4.7625, 7.7258);



  \path[draw=black,miter limit=10.0] (2.1167, 6.9321) -- (2.1167, 5.9267);



  \path[draw=black,fill=white] (0.5292, 8.5196) rectangle (3.7042, 6.9321);



  \begin{scope}[shift={(-0.0132, 0.0132)}]
    \node[text=black,anchor=south,fit={(0,0) (3.175, 1.5875)}] (text8417) at (2.1167, 7.62){aver /= MxM};



  \end{scope}
  \path (4.7625, 8.5328) rectangle (5.4504, 6.9189);



  \path[draw=black,miter limit=10.0] (5.4504, 8.5328) -- (4.7625, 8.5328) -- (4.7625, 6.9189) -- (5.4504, 6.9189);



  \begin{scope}[shift={(-0.0132, 0.0132)}]
    \node[text=black,anchor=south,fit={(0,0) (3.175, 1.5875)} ] (text3422) at (6.5, 7.62){Среднее арифметическое значение в диапазоне [0, 1]};



  \end{scope}
  \path (4.7625, 6.1913) rectangle (5.4504, 4.0481);



  \path[draw=black,miter limit=10.0] (5.4504, 6.1913) -- (4.7625, 6.1913) -- (4.7625, 4.0481) -- (5.4504, 4.0481);



  \begin{scope}[shift={(-0.0132, 0.0132)}]
    \node[text=black,anchor=south,fit={(0,0) (5, 1.5875)},scale=0.7] (text1916) at (6.7, 5.0271){Освещение применяется с учетом значения aver, на которое умножается диффузная и спекулярная состовляющие};



  \end{scope}
  \path[draw=black,miter limit=10.0,dash pattern=on 0.0794cm off 0.0794cm] (3.7042, 5.1313) -- (4.2333, 5.1313) -- (4.7625, 5.1189);



  \path[draw=black,miter limit=10.0] (2.1167, 4.3392) -- (2.1167, 3.2544);



  \path[draw=black,fill=white] (0.5292, 5.9267) rectangle (3.7042, 4.3392);



  \path[draw=black,miter limit=10.0] (0.8467, 5.9267) -- (0.8467, 4.3392)(3.3867, 5.9267) -- (3.3867, 4.3392);



  \begin{scope}[shift={(-0.0132, 0.0132)}]
    \node[text=black,anchor=south,fit={(0,0) (3.175, 1.5875)}] (text9174) at (2.1167, 5.0271){Применить\\полное\\освещение};



  \end{scope}
  \path[draw=black,miter limit=10.0] (14.2875, 0.6085) -- (14.2875, 0.0794) -- (8.6254, 0.0794) -- (8.6254, 9.6044) -- (2.1167, 9.6044) -- (2.1167, 8.5196);



  \path[draw=black,fill=white,miter limit=10.0,cm={ -1.0,-0.0,0.0,-1.0,(28.575, 2.54)}] (13.2292, 1.9315) -- (15.3458, 1.9315) -- (15.875, 1.5081) -- (15.875, 0.6085) -- (12.7, 0.6085) -- (12.7, 1.5081) -- cycle;



  \begin{scope}[shift={(-0.0132, 0.0132)}]
    \node[text=black,anchor=south,fit={(0,0) (3.175, 1.5875)}] (text14) at (14.2875, 1.1642){Пока остались необработанные пиксели};



  \end{scope}

\end{tikzpicture}

\caption{Схема алгоритма теневых карт с линейной фильтрацией (PCF)}
\label{chart:shadow_map_pcf}

\end{figure}

\FloatBarrier
\begin{figure}
  \centering
  \begin{tikzpicture}[y=1cm, x=1cm, yscale=\globalscale*0.8,xscale=\globalscale*0.8, every node/.style={font=\scriptsize}, every node/.append style={scale=\globalscale}, inner sep=0pt, outer sep=0pt]
    \path[draw=black,miter limit=10.0,dash pattern=on 0.0794cm off 0.0794cm] (16.4042, 21.2148) -- (16.6688, 21.2148) -- (16.9333, 21.2148);
  
  
  
    \path[draw=black,fill=white,dash pattern=on 0.0794cm off 0.0794cm] (12.1708, 23.3249) rectangle (16.4042, 19.1048);
  
  
  
    \path[draw=black,miter limit=10.0,dash pattern=on 0.0794cm off 0.0794cm] (4.2333, 15.9298) -- (4.7625, 15.9165);
  
  
  
    \path[draw=black,fill=white,dash pattern=on 0.0794cm off 0.0794cm] (0.0, 21.2214) rectangle (4.2333, 10.6381);
  
  
  
    \path[draw=black,miter limit=10.0,dash pattern=on 0.0794cm off 0.0794cm] (3.7042, 22.2533) -- (4.7625, 22.2533);
  
  
  
    \path[draw=black,miter limit=10.0] (2.1167, 21.4596) -- (2.1167, 20.9304);
  
  
  
    \path[draw=black,fill=white,rounded corners=0.4445cm] (0.5292, 23.0471) rectangle (3.7042, 21.4596);
  
  
  
    \begin{scope}[shift={(-0.0132, 0.0132)}]
      \node[text=black,anchor=south,fit={(0,0) (3.175, 1.5875)}] (text830) at (2.1167, 22.1456){Начало};
  
  
  
    \end{scope}
    \path (4.7625, 23.5762) rectangle (5.7944, 20.9304);
  
  
  
    \path[draw=black,miter limit=10.0] (5.7944, 23.5762) -- (4.7625, 23.5762) -- (4.7625, 20.9304) -- (5.7944, 20.9304);
  
  
  
    \begin{scope}[shift={(-0.0132, 0.0132)}]
      \node[text=black,anchor=south,fit={(0,0) (3.175, 1.5875)} ] (text9586) at (6.5, 22.1456){Алгоритм теневых карт с фильтрацией шумом};
  
  
  
    \end{scope}
    \path[draw=black,miter limit=10.0] (2.1167, 19.3429) -- (2.1167, 18.827);
  
  
  
    \path[draw=black,fill=white] (0.5292, 20.9304) rectangle (3.7042, 19.3429);
  
  
  
    \begin{scope}[shift={(-0.0132, 0.0132)}]
      \node[text=black,anchor=south,fit={(0,0) (3.175, 1.5875)}] (text741) at (2.1167, 20.029){Инициализировать теневую карту};
  
  
  
    \end{scope}
    \path[draw=black,miter limit=10.0] (2.1167, 17.2395) -- (2.1167, 16.7235);
  
  
  
    \path[draw=black,fill=white] (0.5292, 18.827) rectangle (3.7042, 17.2395);
  
  
  
    \begin{scope}[shift={(-0.0132, 0.0132)}]
      \node[text=black,anchor=south,fit={(0,0) (3.175, 1.5875)}] (text7418) at (2.1167, 17.9387){Инициализировать источник света};
  
  
  
    \end{scope}
    \path[draw=black,miter limit=10.0] (2.1167, 15.136) -- (2.1167, 14.6069);
  
  
  
    \path[draw=black,fill=white] (0.5292, 16.7235) rectangle (3.7042, 15.136);
  
  
  
    \begin{scope}[shift={(-0.0132, 0.0132)}]
      \node[text=black,anchor=south,fit={(0,0) (3.175, 1.5875)}] (text6602) at (2.1167, 15.8221){Инициализировать камеру};
  
  
  
    \end{scope}
    \path[draw=black,miter limit=10.0] (2.1167, 13.0194) -- (2.1167, 12.4902);
  
  
  
    \path[draw=black,fill=white] (0.5292, 14.6069) rectangle (3.7042, 13.0194);
  
  
  
    \begin{scope}[shift={(-0.0132, 0.0132)}]
      \node[text=black,anchor=south,fit={(0,0) (3.175, 1.5875)}] (text9819) at (2.1167, 13.7054){Инициализировать модели};
  
  
  
    \end{scope}
    \path (4.7625, 17.4662) rectangle (5.7944, 14.3669);
  
  
  
    \path[draw=black,miter limit=10.0] (5.7944, 17.4662) -- (4.7625, 17.4662) -- (4.7625, 14.3669) -- (5.7944, 14.3669);
  
  
  
    \begin{scope}[shift={(-0.0132, 0.0132)}]
      \node[text=black,anchor=south,fit={(0,0) (3.175, 1.5875)} ] (text7887) at (6.5, 15.8221){Подготовительный этап, инициализирующий минимально необходимые};
  
  
  
    \end{scope}
    \path[draw=black,miter limit=10.0] (2.1167, 10.9027) -- (2.1167, 10.3701) -- (8.6254, 10.3701) -- (8.6254, 23.5992) -- (14.2875, 23.5992) -- (14.2875, 23.0471);
  
  
  
    \path[draw=black,fill=white] (0.5292, 12.4902) rectangle (3.7042, 10.9027);
  
  
  
    \begin{scope}[shift={(-0.0132, 0.0132)}]
      \node[text=black,anchor=south,fit={(0,0) (3.175, 1.5875)}] (text6833) at (2.1167, 11.5887){Инициализировать
  шейдеры};
  
  
  
    \end{scope}
    \path[draw=black,miter limit=10.0] (14.2875, 21.4596) -- (14.2875, 20.9304);
  
  
  
    \path[draw=black,fill=white] (12.7, 23.0471) rectangle (15.875, 21.4596);
  
  
  
    \begin{scope}[shift={(-0.0132, 0.0132)}]
      \node[text=black,anchor=south,fit={(0,0) (3.175, 1.5875)}] (text7755) at (14.2875, 22.1456){Нарисовать сцену с точки зрения источника света};
  
  
  
    \end{scope}
    \path[draw=black,miter limit=10.0] (14.2875, 19.3429) -- (14.2875, 18.5756);
  
  
  
    \path[draw=black,fill=white] (12.7, 20.9304) rectangle (15.875, 19.3429);
  
  
  
    \begin{scope}[shift={(-0.0132, 0.0132)}]
      \node[text=black,anchor=south,fit={(0,0) (3.175, 1.5875)}] (text3051) at (14.2875, 20.029){Заполнить теневую карту значениями глубины};
  
  
  
    \end{scope}
    \path (16.9333, 22.5377) rectangle (17.9652, 19.8919);
  
  
  
    \path[draw=black,miter limit=10.0] (17.9652, 22.5377) -- (16.9333, 22.5377) -- (16.9333, 19.8919) -- (17.9652, 19.8919);
  
  
  
    \begin{scope}[shift={(-0.0132, 0.0132)}]
      \node[text=black,anchor=south,fit={(0,0) (3.175, 1.5875)} ] (text4923) at (18.5, 21.1137){Этап заполнения теневой карты};
  
  
  
    \end{scope}
    \path[draw=black,miter limit=10.0] (14.2875, 17.2527) -- (14.2875, 16.7235);
  
  
  
    \path[draw=black,fill=white,miter limit=10.0] (13.2292, 18.5756) -- (15.3458, 18.5756) -- (15.875, 18.1523) -- (15.875, 17.2527) -- (12.7, 17.2527) -- (12.7, 18.1523) -- cycle;
  
  
  
    \begin{scope}[shift={(-0.0132, 0.0132)}]
      \node[text=black,anchor=south,fit={(0,0) (3.175, 1.5875)}] (text4800) at (14.2875, 17.8065){Обработка N пикселей};
  
  
  
    \end{scope}
    \path[draw=black,miter limit=10.0,dash pattern=on 0.0794cm off 0.0794cm] (15.875, 15.9298) -- (16.6688, 15.9298);
  
  
  
    \path[draw=black,miter limit=10.0] (14.2875, 15.136) -- (14.2875, 14.6069);
  
  
  
    \path[draw=black,fill=white] (12.7, 16.7235) rectangle (15.875, 15.136);
  
  
  
    \begin{scope}[shift={(-0.0132, 0.0132)}]
      \node[text=black,anchor=south,fit={(0,0) (3.175, 1.5875)}] (text1572) at (14.2875, 15.8221){Преобразовать координаты фрагмента в пространство света};
  
  
  
    \end{scope}
    \path (16.6688, 16.7367) rectangle (17.7271, 15.1228);
  
  
  
    \path[draw=black,miter limit=10.0] (17.7271, 16.7367) -- (16.6688, 16.7367) -- (16.6688, 15.1228) -- (17.7271, 15.1228);
  
  
  
    \begin{scope}[shift={(-0.0132, 0.0132)}]
      \node[text=black,anchor=south,fit={(0,0) (3.175, 1.5875)} ] (text2822) at (18, 15.8221){Далее
  d1};
  
  
  
    \end{scope}
    \path[draw=black,fill=white,rounded corners=0.4445cm] (0.5292, 3.2544) rectangle (3.7042, 1.6669);
  
  
  
    \begin{scope}[shift={(-0.0132, 0.0132)}]
      \node[text=black,anchor=south,fit={(0,0) (3.175, 1.5875)}] (text75) at (2.1167, 2.3548){Конец};
  
  
  
    \end{scope}
    \path[draw=black,miter limit=10.0] (14.2875, 11.1673) -- (14.2875, 10.6381);
  
  
  
    \path[draw=black,fill=white,miter limit=10.0] (13.2292, 12.4902) -- (15.3458, 12.4902) -- (15.875, 12.0669) -- (15.875, 11.1673) -- (12.7, 11.1673) -- (12.7, 12.0669) -- cycle;
  
  
  
    \begin{scope}[shift={(-0.0132, 0.0132)}]
      \node[text=black,anchor=south,fit={(0,0) (3.175, 1.5875)},scale=0.9] (text4690) at (14.2875, 11.721){Обработка MxM окаймляющих пикселей};
  
  
  
    \end{scope}
    \path[draw=black,miter limit=10.0] (14.2875, 13.0194) -- (14.2875, 12.4902);
  
  
  
    \path[draw=black,miter limit=10.0,dash pattern=on 0.0794cm off 0.0794cm] (15.875, 13.806) -- (16.6688, 13.8023);
  
  
  
    \path[draw=black,fill=white] (12.7, 14.6069) rectangle (15.875, 13.0194);
  
  
  
    \begin{scope}[shift={(-0.0132, 0.0132)}]
      \node[text=black,anchor=south,fit={(0,0) (3.175, 1.5875)}] (text5632) at (14.2875, 13.7054){aver = 0};
  
  
  
    \end{scope}
    \path[draw=black,miter limit=10.0,dash pattern=on 0.0794cm off 0.0794cm] (15.875, 9.8515) -- (16.6688, 9.8552);
  
  
  
    \path[draw=black,miter limit=10.0] (14.2875, 9.0506) -- (14.2875, 8.5196);
  
  
  
    \path[draw=black,fill=white] (12.7, 10.6381) rectangle (15.875, 9.0506);
  
  
  
    \begin{scope}[shift={(-0.0132, 0.0132)}]
      \node[text=black,anchor=south,fit={(0,0) (3.175, 1.5875)},scale=0.9] (text8315) at (14.2875, 9.7367){Считать соответствующую глубину в окрестности со сдвигом по шуму};
  
  
  
    \end{scope}
    \path (16.6688, 10.6646) rectangle (17.7271, 9.0506);
  
  
  
    \path[draw=black,miter limit=10.0] (17.7271, 10.6646) -- (16.6688, 10.6646) -- (16.6688, 9.0506) -- (17.7271, 9.0506);
  
  
  
    \begin{scope}[shift={(-0.0132, 0.0132)}]
      \node[text=black,anchor=south,fit={(0,0) (3.175, 1.5875)} ] (text240) at (18, 9.7631){Далее
  d2};
  
  
  
    \end{scope}
    \path (16.6688, 14.6069) rectangle (17.7271, 12.9929);
  
  
  
    \path[draw=black,miter limit=10.0] (17.7271, 14.6069) -- (16.6688, 14.6069) -- (16.6688, 12.9929) -- (17.7271, 12.9929);
  
  
  
    \begin{scope}[shift={(-0.0132, 0.0132)}]
      \node[text=black,anchor=south,fit={(0,0) (3.175, 1.5875)} ] (text3334) at (18.5, 13.7054){Количество перекрываемых глубин};
  
  
  
    \end{scope}
    \path[draw=black,miter limit=10.0] (15.875, 7.7258) -- (16.3761, 7.7242) -- (16.3761, 6.5979);
  
  
  
    \path[draw=black,fill=black,miter limit=10.0] (16.3761, 6.459) -- (16.2835, 6.6442) -- (16.3761, 6.5979) -- (16.4687, 6.6442) -- cycle;
  
  
  
    \begin{scope}[shift={(-0.0132, 0.0132)}]
      \node[text=black,anchor=south,fit={(0,0) (3.175, 1.5875)}] (text4706) at (16.0602, 7.911){Да};
  
  
  
    \end{scope}
    \path[draw=black,miter limit=10.0] (12.7, 7.7258) -- (12.065, 7.7242) -- (12.065, 4.3127) -- (14.2875, 4.3127) -- (14.2875, 3.9521);
  
  
  
    \path[draw=black,fill=black,miter limit=10.0] (14.2875, 3.8132) -- (14.1949, 3.9984) -- (14.2875, 3.9521) -- (14.3801, 3.9984) -- cycle;
  
  
  
    \path[draw=black,fill=white,miter limit=10.0] (14.2875, 8.5196) -- (15.875, 7.7258) -- (14.2875, 6.9321) -- (12.7, 7.7258) -- cycle;
  
  
  
    \begin{scope}[shift={(-0.0132, 0.0132)}]
      \node[text=black,anchor=south,fit={(0,0) (3.175, 1.5875)}] (text9537) at (14.2875, 7.62){d1 > d2};
  
  
  
    \end{scope}
    \path[draw=black,miter limit=10.0] (16.3761, 4.8419) -- (16.3761, 4.3127) -- (14.2875, 4.3127) -- (14.2875, 3.9521);
  
  
  
    \path[draw=black,fill=black,miter limit=10.0] (14.2875, 3.8132) -- (14.1949, 3.9984) -- (14.2875, 3.9521) -- (14.3801, 3.9984) -- cycle;
  
  
  
    \path[draw=black,fill=white] (14.7902, 6.4294) rectangle (17.9652, 4.8419);
  
  
  
    \begin{scope}[shift={(-0.0132, 0.0132)}]
      \node[text=black,anchor=south,fit={(0,0) (3.175, 1.5875)}] (text395) at (16.3777, 5.5298){aver += 1};
  
  
  
    \end{scope}
    \path[draw=black,miter limit=10.0] (14.2875, 2.4606) -- (14.2875, 1.9315);
  
  
  
    \path[draw=black,fill=white,miter limit=10.0,cm={ -1.0,-0.0,0.0,-1.0,(28.575, 6.2442)}] (13.2292, 3.7835) -- (15.3458, 3.7835) -- (15.875, 3.3602) -- (15.875, 2.4606) -- (12.7, 2.4606) -- (12.7, 3.3602) -- cycle;
  
  
  
    \begin{scope}[shift={(-0.0132, 0.0132)}]
      \node[text=black,anchor=south,fit={(0,0) (3.175, 1.5875)}] (text193) at (14.2875, 3.0162){Пока остались пиксели в окрестности};
  
  
  
    \end{scope}
    \path[draw=black,miter limit=10.0,dash pattern=on 0.0794cm off 0.0794cm] (3.7042, 7.7258) -- (4.7625, 7.7258);
  
  
  
    \path[draw=black,miter limit=10.0] (2.1167, 6.9321) -- (2.1167, 5.9267);
  
  
  
    \path[draw=black,fill=white] (0.5292, 8.5196) rectangle (3.7042, 6.9321);
  
  
  
    \begin{scope}[shift={(-0.0132, 0.0132)}]
      \node[text=black,anchor=south,fit={(0,0) (3.175, 1.5875)}] (text8417) at (2.1167, 7.62){aver /= MxM};
  
  
  
    \end{scope}
    \path (4.7625, 8.5328) rectangle (5.4504, 6.9189);
  
  
  
    \path[draw=black,miter limit=10.0] (5.4504, 8.5328) -- (4.7625, 8.5328) -- (4.7625, 6.9189) -- (5.4504, 6.9189);
  
  
  
    \begin{scope}[shift={(-0.0132, 0.0132)}]
      \node[text=black,anchor=south,fit={(0,0) (3.175, 1.5875)} ] (text3422) at (6.5, 7.62){Среднее арифметическое значение в диапазоне [0, 1]};
  
  
  
    \end{scope}
    \path (4.7625, 6.1913) rectangle (5.4504, 4.0481);
  
  
  
    \path[draw=black,miter limit=10.0] (5.4504, 6.1913) -- (4.7625, 6.1913) -- (4.7625, 4.0481) -- (5.4504, 4.0481);
  
  
  
    \begin{scope}[shift={(-0.0132, 0.0132)}]
      \node[text=black,anchor=south,fit={(0,0) (5, 1.5875)},scale=0.7] (text1916) at (6.7, 5.0271){Освещение применяется с учетом значения aver, на которое умножается диффузная и спекулярная состовляющие};
  
  
  
    \end{scope}
    \path[draw=black,miter limit=10.0,dash pattern=on 0.0794cm off 0.0794cm] (3.7042, 5.1313) -- (4.2333, 5.1313) -- (4.7625, 5.1189);
  
  
  
    \path[draw=black,miter limit=10.0] (2.1167, 4.3392) -- (2.1167, 3.2544);
  
  
  
    \path[draw=black,fill=white] (0.5292, 5.9267) rectangle (3.7042, 4.3392);
  
  
  
    \path[draw=black,miter limit=10.0] (0.8467, 5.9267) -- (0.8467, 4.3392)(3.3867, 5.9267) -- (3.3867, 4.3392);
  
  
  
    \begin{scope}[shift={(-0.0132, 0.0132)}]
      \node[text=black,anchor=south,fit={(0,0) (3.175, 1.5875)}] (text9174) at (2.1167, 5.0271){Применить\\полное\\освещение};
  
  
  
    \end{scope}
    \path[draw=black,miter limit=10.0] (14.2875, 0.6085) -- (14.2875, 0.0794) -- (8.6254, 0.0794) -- (8.6254, 9.6044) -- (2.1167, 9.6044) -- (2.1167, 8.5196);
  
  
  
    \path[draw=black,fill=white,miter limit=10.0,cm={ -1.0,-0.0,0.0,-1.0,(28.575, 2.54)}] (13.2292, 1.9315) -- (15.3458, 1.9315) -- (15.875, 1.5081) -- (15.875, 0.6085) -- (12.7, 0.6085) -- (12.7, 1.5081) -- cycle;
  
  
  
    \begin{scope}[shift={(-0.0132, 0.0132)}]
      \node[text=black,anchor=south,fit={(0,0) (3.175, 1.5875)}] (text14) at (14.2875, 1.1642){Пока остались необработанные пиксели};
  
  
  
    \end{scope}
  
  \end{tikzpicture}
  
  \caption{Схема алгоритма теневых карт с линейной фильтрацией шумом (NOISE)}
  \label{chart:shadow_map_noise}
  
  \end{figure}
  
\FloatBarrier
\begin{figure}
\centering
\begin{tikzpicture}[y=1cm, x=1cm, yscale=\globalscale*0.74,xscale=\globalscale*0.74, every node/.style={font=\scriptsize}, every node/.append style={scale=\globalscale}, inner sep=0pt, outer sep=0pt]
  \path[draw=black,miter limit=10.0,dash pattern=on 0.0794cm off 0.0794cm] (13.1233, 19.9708) -- (13.3879, 19.9708) -- (13.6525, 19.9713);



  \path[draw=black,fill=white,dash pattern=on 0.0794cm off 0.0794cm] (8.89, 22.0813) rectangle (13.1233, 17.8612);



  \path[draw=black,miter limit=10.0,dash pattern=on 0.0794cm off 0.0794cm] (4.2333, 14.6598) -- (4.7625, 14.6465);



  \path[draw=black,fill=white,dash pattern=on 0.0794cm off 0.0794cm] (0.0, 19.9514) rectangle (4.2333, 9.3681);



  \path[draw=black,miter limit=10.0,dash pattern=on 0.0794cm off 0.0794cm] (3.7042, 20.9833) -- (4.7625, 20.9833);



  \path[draw=black,miter limit=10.0] (2.1167, 20.1896) -- (2.1167, 19.6604);



  \path[draw=black,fill=white,rounded corners=0.4445cm] (0.5292, 21.7771) rectangle (3.7042, 20.1896);



  \begin{scope}[shift={(-0.0132, 0.0132)}]
    \node[text=black,anchor=south,fit={(0,0) (3.175, 1.5875)}] (text5442) at (2.1167, 20.8756){Начало};



  \end{scope}
  \path (4.7625, 22.3062) rectangle (5.7944, 19.6604);



  \path[draw=black,miter limit=10.0] (5.7944, 22.3062) -- (4.7625, 22.3062) -- (4.7625, 19.6604) -- (5.7944, 19.6604);



  \begin{scope}[shift={(-0.0132, 0.0132)}]
    \node[text=black,anchor=south,fit={(0,0) (3, 1.5875)},scale=1] (text5883) at (6.2, 20.8756){Алгоритм теневых карт с фильтрацией (PCF)};



  \end{scope}
  \path[draw=black,miter limit=10.0] (2.1167, 18.0729) -- (2.1167, 17.557);



  \path[draw=black,fill=white] (0.5292, 19.6604) rectangle (3.7042, 18.0729);



  \begin{scope}[shift={(-0.0132, 0.0132)}]
    \node[text=black,anchor=south,fit={(0,0) (3.175, 1.5875)}] (text1439) at (2.1167, 18.759){Инициализировать теневую карту};



  \end{scope}
  \path[draw=black,miter limit=10.0] (2.1167, 15.9695) -- (2.1167, 15.4535);



  \path[draw=black,fill=white] (0.5292, 17.557) rectangle (3.7042, 15.9695);



  \begin{scope}[shift={(-0.0132, 0.0132)}]
    \node[text=black,anchor=south,fit={(0,0) (3.175, 1.5875)}] (text4655) at (2.1167, 16.6688){Инициализировать источник света};



  \end{scope}
  \path[draw=black,miter limit=10.0] (2.1167, 13.866) -- (2.1167, 13.3369);



  \path[draw=black,fill=white] (0.5292, 15.4535) rectangle (3.7042, 13.866);



  \begin{scope}[shift={(-0.0132, 0.0132)}]
    \node[text=black,anchor=south,fit={(0,0) (3.175, 1.5875)}] (text9135) at (2.1167, 14.5521){Инициализировать камеру};



  \end{scope}
  \path[draw=black,miter limit=10.0] (2.1167, 11.7494) -- (2.1167, 11.2202);



  \path[draw=black,fill=white] (0.5292, 13.3369) rectangle (3.7042, 11.7494);



  \begin{scope}[shift={(-0.0132, 0.0132)}]
    \node[text=black,anchor=south,fit={(0,0) (3.175, 1.5875)}] (text9597) at (2.1167, 12.4354){Инициализировать модели};



  \end{scope}
  \path (4.7625, 16.1962) rectangle (5.7944, 13.0969);



  \path[draw=black,miter limit=10.0] (5.7944, 16.1962) -- (4.7625, 16.1962) -- (4.7625, 13.0969) -- (5.7944, 13.0969);



  \begin{scope}[shift={(-0.0132, 0.0132)}]
    \node[text=black,anchor=south,fit={(0,0) (3.175, 1.5875)},scale=0.9] (text121) at (6.35, 14.5521){Подготовительный этап, инициализирующий минимально необходимые данные};



  \end{scope}
  \path[draw=black,miter limit=10.0] (2.1167, 9.6327) -- (2.1167, 9.1017) -- (8.0963, 9.1017) -- (8.0963, 22.352) -- (11.0067, 22.352) -- (11.0067, 21.8035);



  \path[draw=black,fill=white] (0.5292, 11.2202) rectangle (3.7042, 9.6327);



  \begin{scope}[shift={(-0.0132, 0.0132)}]
    \node[text=black,anchor=south,fit={(0,0) (3.175, 1.5875)}] (text1075) at (2.1167, 10.3188){Инициализировать
шейдеры};



  \end{scope}
  \path[draw=black,miter limit=10.0] (11.0067, 20.216) -- (11.0067, 19.6869);



  \path[draw=black,fill=white] (9.4192, 21.8035) rectangle (12.5942, 20.216);



  \begin{scope}[shift={(-0.0132, 0.0132)}]
    \node[text=black,anchor=south,fit={(0,0) (3.175, 1.5875)}] (text8288) at (11.0067, 20.9021){Нарисовать сцену с точки зрения источника света};



  \end{scope}
  \path[draw=black,miter limit=10.0] (11.0067, 18.0994) -- (11.0067, 17.3321);



  \path[draw=black,fill=white] (9.4192, 19.6869) rectangle (12.5942, 18.0994);



  \begin{scope}[shift={(-0.0132, 0.0132)}]
    \node[text=black,anchor=south,fit={(0,0) (3.175, 1.5875)}] (text6434) at (11.0067, 18.7854){Заполнить теневую карту значениями глубины};



  \end{scope}
  \path (13.6525, 21.2942) rectangle (14.6844, 18.6484);



  \path[draw=black,miter limit=10.0] (14.6844, 21.2942) -- (13.6525, 21.2942) -- (13.6525, 18.6484) -- (14.6844, 18.6484);



  \begin{scope}[shift={(-0.0132, 0.0132)}]
    \node[text=black,anchor=south,fit={(0,0) (2, 1.5875)} ] (text9891) at (14.75, 19.8702){Этап заполнения теневой карты};



  \end{scope}
  \path[draw=black,miter limit=10.0] (11.0067, 16.0091) -- (11.0067, 15.48);



  \path[draw=black,fill=white,miter limit=10.0] (9.9483, 17.3321) -- (12.065, 17.3321) -- (12.5942, 16.9087) -- (12.5942, 16.0091) -- (9.4192, 16.0091) -- (9.4192, 16.9087) -- cycle;



  \begin{scope}[shift={(-0.0132, 0.0132)}]
    \node[text=black,anchor=south,fit={(0,0) (3.175, 1.5875)}] (text5014) at (11.0067, 16.5629){Обработка N пикселей};



  \end{scope}
  \path[draw=black,miter limit=10.0,dash pattern=on 0.0794cm off 0.0794cm] (12.5942, 14.6862) -- (13.3879, 14.6862);



  \path[draw=black,miter limit=10.0] (11.0067, 13.8925) -- (11.0067, 13.3633);



  \path[draw=black,fill=white] (9.4192, 15.48) rectangle (12.5942, 13.8925);



  \begin{scope}[shift={(-0.0132, 0.0132)}]
    \node[text=black,anchor=south,fit={(0,0) (3.175, 1.5875)},scale=0.8] (text8939) at (11.0067, 14.5785){Преобразовать координаты фрагмента в пространство света};



  \end{scope}
  \path (13.3879, 15.4932) rectangle (14.4463, 13.8792);



  \path[draw=black,miter limit=10.0] (14.4463, 15.4932) -- (13.3879, 15.4932) -- (13.3879, 13.8792) -- (14.4463, 13.8792);



  \begin{scope}[shift={(-0.0132, 0.0132)}]
    \node[text=black,anchor=south,fit={(0,0) (3.175, 1.5875)} ] (text9548) at (14.5, 14.5785){Далее
d1};



  \end{scope}
  \path[draw=black,fill=white,rounded corners=0.4445cm] (0.5292, 1.9844) rectangle (3.7042, 0.3969);



  \begin{scope}[shift={(-0.0132, 0.0132)}]
    \node[text=black,anchor=south,fit={(0,0) (3.175, 1.5875)}] (text5062) at (2.1167, 1.0848){Конец};



  \end{scope}
  \path[draw=black,miter limit=10.0] (11.0067, 9.9237) -- (11.0067, 9.3946);



  \path[draw=black,fill=white,miter limit=10.0] (9.9483, 11.2466) -- (12.065, 11.2466) -- (12.5942, 10.8233) -- (12.5942, 9.9237) -- (9.4192, 9.9237) -- (9.4192, 10.8233) -- cycle;



  \begin{scope}[shift={(-0.0132, 0.0132)}]
    \node[text=black,anchor=south,fit={(0,0) (3.175, 1.5875)},scale=0.9] (text1678) at (11.0067, 10.4775){Обработка S блокирующих, окаймляющий пикселей};



  \end{scope}
  \path[draw=black,miter limit=10.0] (11.0067, 11.7758) -- (11.0067, 11.2466);



  \path[draw=black,miter limit=10.0,dash pattern=on 0.0794cm off 0.0794cm] (12.5942, 12.5624) -- (13.3879, 12.5587);



  \path[draw=black,fill=white] (9.4192, 13.3633) rectangle (12.5942, 11.7758);



  \begin{scope}[shift={(-0.0132, 0.0132)}]
    \node[text=black,anchor=south,fit={(0,0) (3.175, 1.5875)}] (text1321) at (11.0067, 12.4619){sumBlocker = 0};



  \end{scope}
  \path[draw=black,miter limit=10.0,dash pattern=on 0.0794cm off 0.0794cm] (12.5942, 8.608) -- (13.3879, 8.6117);



  \path[draw=black,miter limit=10.0] (11.0067, 7.8071) -- (11.0067, 7.276);



  \path[draw=black,fill=white] (9.4192, 9.3946) rectangle (12.5942, 7.8071);



  \begin{scope}[shift={(-0.0132, 0.0132)}]
    \node[text=black,anchor=south,fit={(0,0) (3.175, 1.5875)}] (text1824) at (11.0067, 8.4931){Считать соответствующую глубину в окрестности};



  \end{scope}
  \path (13.3879, 9.421) rectangle (14.4463, 7.8071);



  \path[draw=black,miter limit=10.0] (14.4463, 9.421) -- (13.3879, 9.421) -- (13.3879, 7.8071) -- (14.4463, 7.8071);



  \begin{scope}[shift={(-0.0132, 0.0132)}]
    \node[text=black,anchor=south,fit={(0,0) (3.175, 1.5875)} ] (text5006) at (14.5, 8.5196){Далее
d2};



  \end{scope}
  \path (13.3879, 13.3633) rectangle (14.4463, 11.7494);



  \path[draw=black,miter limit=10.0] (14.4463, 13.3633) -- (13.3879, 13.3633) -- (13.3879, 11.7494) -- (14.4463, 11.7494);



  \begin{scope}[shift={(-0.0132, 0.0132)}]
    \node[text=black,anchor=south,fit={(0,0) (3.175, 1.5875)},scale=0.9] (text3773) at (14.5, 12.4619){Количество блокирующих глубин};



  \end{scope}
  \path[draw=black,miter limit=10.0] (12.5942, 6.4823) -- (13.1022, 6.477) -- (13.1022, 5.3544);



  \path[draw=black,fill=black,miter limit=10.0] (13.1022, 5.2155) -- (13.0096, 5.4007) -- (13.1022, 5.3544) -- (13.1948, 5.4007) -- cycle;



  \begin{scope}[shift={(-0.0132, 0.0132)}]
    \node[text=black,anchor=south,fit={(0,0) (3.175, 1.5875)}] (text4912) at (12.8323, 6.641){Да};



  \end{scope}
  \path[draw=black,miter limit=10.0] (9.4192, 6.4823) -- (8.7842, 6.477) -- (8.7842, 3.0692) -- (11.0067, 3.0692) -- (11.0067, 2.7085);



  \path[draw=black,fill=black,miter limit=10.0] (11.0067, 2.5696) -- (10.9141, 2.7548) -- (11.0067, 2.7085) -- (11.0993, 2.7548) -- cycle;



  \path[draw=black,fill=white,miter limit=10.0] (11.0067, 7.276) -- (12.5942, 6.4823) -- (11.0067, 5.6885) -- (9.4192, 6.4823) -- cycle;



  \begin{scope}[shift={(-0.0132, 0.0132)}]
    \node[text=black,anchor=south,fit={(0,0) (3.175, 1.5875)}] (text3271) at (11.0067, 6.3765){d1 > d2};



  \end{scope}
  \path[draw=black,miter limit=10.0] (13.1022, 3.5983) -- (13.1022, 3.0692) -- (11.0067, 3.0692) -- (11.0067, 2.7085);



  \path[draw=black,fill=black,miter limit=10.0] (11.0067, 2.5696) -- (10.9141, 2.7548) -- (11.0067, 2.7085) -- (11.0993, 2.7548) -- cycle;



  \path[draw=black,fill=white] (11.5094, 5.1858) rectangle (14.6844, 3.5983);



  \begin{scope}[shift={(-0.0132, 0.0132)}]
    \node[text=black,anchor=south,fit={(0,0) (3.175, 1.5875)}] (text3007) at (13.0969, 4.2862){sumBlocker += 1};



  \end{scope}
  \path[draw=black,miter limit=10.0] (11.0067, 1.2171) -- (11.0067, 0.6562) -- (16.0338, 0.6562) -- (16.0338, 22.352) -- (18.6796, 22.352) -- (18.6796, 21.8167);



  \path[draw=black,fill=white,miter limit=10.0,cm={ -1.0,-0.0,0.0,-1.0,(22.0133, 3.7571)}] (9.9483, 2.54) -- (12.065, 2.54) -- (12.5942, 2.1167) -- (12.5942, 1.2171) -- (9.4192, 1.2171) -- (9.4192, 2.1167) -- cycle;



  \begin{scope}[shift={(-0.0132, 0.0132)}]
    \node[text=black,anchor=south,fit={(0,0) (3.175, 1.5875)},scale=0.9] (text4100) at (11.0067, 1.7727){Пока остались  блокирующие пиксели};



  \end{scope}
  \path[draw=black,miter limit=10.0,dash pattern=on 0.0794cm off 0.0794cm] (3.7042, 6.4558) -- (4.7625, 6.4558);



  \path[draw=black,miter limit=10.0] (2.1167, 5.6621) -- (2.1167, 4.6567);



  \path[draw=black,fill=white] (0.5292, 7.2496) rectangle (3.7042, 5.6621);



  \begin{scope}[shift={(-0.0132, 0.0132)}]
    \node[text=black,anchor=south,fit={(0,0) (3.175, 1.5875)}] (text6853) at (2.1167, 6.35){aver /= MxM};



  \end{scope}
  \path (4.7625, 7.2628) rectangle (5.4504, 5.6489);



  \path[draw=black,miter limit=10.0] (5.4504, 7.2628) -- (4.7625, 7.2628) -- (4.7625, 5.6489) -- (5.4504, 5.6489);



  \begin{scope}[shift={(-0.0132, 0.0132)}]
    \node[text=black,anchor=south,fit={(0,0) (3.175, 1.5875)} ] (text9927) at (6.5, 6.35){Среднее арифметическое значение в диапазоне [0, 1]};



  \end{scope}
  \path (4.7625, 4.9213) rectangle (5.4504, 2.7781);



  \path[draw=black,miter limit=10.0] (5.4504, 4.9213) -- (4.7625, 4.9213) -- (4.7625, 2.7781) -- (5.4504, 2.7781);



  \begin{scope}[shift={(-0.0132, 0.0132)}]
    \node[text=black,anchor=south,fit={(0,0) (4.5, 1.5875)},scale=0.65] (text9191) at (6.5, 3.7571){Освещение применяется с учетом значения aver, на которое умножается диффузная и спекулярная состовляющие};



  \end{scope}
  \path[draw=black,miter limit=10.0,dash pattern=on 0.0794cm off 0.0794cm] (3.7042, 3.8629) -- (4.2333, 3.8629) -- (4.7625, 3.8523);



  \path[draw=black,miter limit=10.0] (2.1167, 3.0692) -- (2.1167, 1.9844);



  \path[draw=black,fill=white] (0.5292, 4.6567) rectangle (3.7042, 3.0692);



  \path[draw=black,miter limit=10.0] (0.8467, 4.6567) -- (0.8467, 3.0692)(3.3867, 4.6567) -- (3.3867, 3.0692);



  \begin{scope}[shift={(-0.0132, 0.0132)}]
    \node[text=black,anchor=south,fit={(0,0) (3.175, 1.5875)}] (text9131) at (2.1167, 3.7571){Применить\\полное\\освещение};



  \end{scope}
  \path[draw=black,miter limit=10.0] (18.6796, 3.5983) -- (18.6796, 0.127) -- (8.0963, 0.127) -- (8.0963, 7.7999) -- (2.1167, 7.7999) -- (2.1167, 7.2496);



  \path[draw=black,fill=white,miter limit=10.0,cm={ -1.0,-0.0,0.0,-1.0,(37.3592, 8.5196)}] (17.6212, 4.9213) -- (19.7379, 4.9213) -- (20.2671, 4.4979) -- (20.2671, 3.5983) -- (17.0921, 3.5983) -- (17.0921, 4.4979) -- cycle;



  \begin{scope}[shift={(-0.0132, 0.0132)}]
    \node[text=black,anchor=south,fit={(0,0) (3.175, 1.5875)}] (text5500) at (18.6796, 4.154){Пока остались необработанные пиксели};



  \end{scope}
  \path[draw=black,miter limit=10.0,dash pattern=on 0.0794cm off 0.0794cm] (20.2671, 21.023) -- (20.7963, 21.023);



  \path[draw=black,miter limit=10.0] (18.6796, 20.2292) -- (18.6796, 19.6869);



  \path[draw=black,fill=white] (17.0921, 21.8167) rectangle (20.2671, 20.2292);



  \begin{scope}[shift={(-0.0132, 0.0132)}]
    \node[text=black,anchor=south,fit={(0,0) (3.175, 1.5875)}] (text8144) at (18.6796, 20.9285){bd = \(\frac{sumBlocker}{S}\)};



  \end{scope}
  \path (20.7963, 21.83) rectangle (21.4842, 20.216);



  \path[draw=black,miter limit=10.0] (21.4842, 21.83) -- (20.7963, 21.83) -- (20.7963, 20.216) -- (21.4842, 20.216);



  \begin{scope}[shift={(-0.0132, 0.0132)}]
    \node[text=black,anchor=south,fit={(0,0) (3.175, 1.5875)},scale=0.8] (text8957) at (22, 20.9285){Среднее арифметическое значение в диапазоне [0, 1]};



  \end{scope}
  \path[draw=black,miter limit=10.0,dash pattern=on 0.0794cm off 0.0794cm] (20.2671, 18.8931) -- (20.7963, 18.8931);



  \path[draw=black,miter limit=10.0] (18.6796, 18.0994) -- (18.6796, 17.8329) -- (18.6796, 17.5966);



  \path[draw=black,fill=white] (17.0921, 19.6869) rectangle (20.2671, 18.0994);



  \begin{scope}[shift={(-0.0132, 0.0132)}]
    \node[text=black,anchor=south,fit={(0,0) (3.175, 1.5875)}] (text9770) at (18.6796, 18.7854){$penumbraSize = \frac{d1 - bd}{bd}$};



  \end{scope}
  \path (20.7963, 19.7001) rectangle (21.4842, 18.0861);



  \path[draw=black,miter limit=10.0] (21.4842, 19.7001) -- (20.7963, 19.7001) -- (20.7963, 18.0861) -- (21.4842, 18.0861);



  \begin{scope}[shift={(-0.0132, 0.0132)}]
    \node[text=black,anchor=south,fit={(0,0) (3.175, 1.5875)},scale=0.75] (text7127) at (22, 18.7854){Расчет размера полутени};



  \end{scope}
  \path[draw=black,miter limit=10.0] (18.6796, 14.1552) -- (18.6796, 13.8853) -- (18.6796, 13.6279);



  \path[draw=black,fill=white,miter limit=10.0] (17.6212, 15.4781) -- (19.7379, 15.4781) -- (20.2671, 15.0548) -- (20.2671, 14.1552) -- (17.0921, 14.1552) -- (17.0921, 15.0548) -- cycle;



  \begin{scope}[shift={(-0.0132, 0.0132)}]
    \node[text=black,anchor=south,fit={(0,0) (3.175, 1.5875)},scale=0.9] (text334) at (18.6796, 14.7108){Обработка MxM окаймляющий пикселей};



  \end{scope}
  \path[draw=black,miter limit=10.0] (18.6796, 16.0091) -- (18.6796, 15.4781);



  \path[draw=black,fill=white] (17.0921, 17.5966) rectangle (20.2671, 16.0091);



  \begin{scope}[shift={(-0.0132, 0.0132)}]
    \node[text=black,anchor=south,fit={(0,0) (3.175, 1.5875)}] (text9090) at (18.6796, 16.6952){aver = 0};



  \end{scope}
  \path[draw=black,miter limit=10.0,dash pattern=on 0.0794cm off 0.0794cm] (20.2671, 12.8341) -- (20.7963, 12.8341);



  \path[draw=black,miter limit=10.0] (18.6785, 12.0404) -- (18.6785, 11.5112);



  \path[draw=black,fill=white] (17.0921, 13.6279) rectangle (20.2671, 12.0404);



  \begin{scope}[shift={(-0.0132, 0.0132)}]
    \node[text=black,anchor=south,fit={(0,0) (3.175, 1.5875)}] (text6779) at (18.6796, 12.7265){Считать соответствующую глубину в окрестности};



  \end{scope}
  \path (20.7963, 13.6411) rectangle (21.4842, 12.0272);



  \path[draw=black,miter limit=10.0] (21.4842, 13.6411) -- (20.7963, 13.6411) -- (20.7963, 12.0272) -- (21.4842, 12.0272);



  \begin{scope}[shift={(-0.0132, 0.0132)}]
    \node[text=black,anchor=south,fit={(0,0) (3.175, 1.5875)},scale=0.75] (text1130) at (22, 12.7265){Окрестность масштабируется в penumbraSize раз. Далее d3};



  \end{scope}
  \path[draw=black,miter limit=10.0] (20.266, 10.7175) -- (20.7751, 10.7209) -- (20.7751, 9.5896);



  \path[draw=black,fill=black,miter limit=10.0] (20.7751, 9.4507) -- (20.6825, 9.6359) -- (20.7751, 9.5896) -- (20.8677, 9.6359) -- cycle;



  \begin{scope}[shift={(-0.0132, 0.0132)}]
    \node[text=black,anchor=south,fit={(0,0) (3.175, 1.5875)}] (text198) at (20.5052, 10.8744){Да};



  \end{scope}
  \path[draw=black,miter limit=10.0] (17.091, 10.7175) -- (16.4571, 10.7209) -- (16.4571, 7.3025) -- (18.6796, 7.3025) -- (18.6796, 6.9154);



  \path[draw=black,fill=black,miter limit=10.0] (18.6796, 6.7765) -- (18.587, 6.9617) -- (18.6796, 6.9154) -- (18.7722, 6.9617) -- cycle;



  \path[draw=black,fill=white,miter limit=10.0] (18.6785, 11.5112) -- (20.266, 10.7175) -- (18.6785, 9.9237) -- (17.091, 10.7175) -- cycle;



  \begin{scope}[shift={(-0.0132, 0.0132)}]
    \node[text=black,anchor=south,fit={(0,0) (3.175, 1.5875)}] (text3298) at (18.6796, 10.6098){d1 > d3};



  \end{scope}
  \path[draw=black,miter limit=10.0] (20.7751, 7.8335) -- (20.7751, 7.3025) -- (18.6796, 7.3025) -- (18.6796, 6.9154);



  \path[draw=black,fill=black,miter limit=10.0] (18.6796, 6.7765) -- (18.587, 6.9617) -- (18.6796, 6.9154) -- (18.7722, 6.9617) -- cycle;



  \path[draw=black,fill=white] (19.1812, 9.421) rectangle (22.3562, 7.8335);



  \begin{scope}[shift={(-0.0132, 0.0132)}]
    \node[text=black,anchor=south,fit={(0,0) (3.175, 1.5875)}] (text2209) at (20.7698, 8.5196){aver += 1};



  \end{scope}
  \path[draw=black,miter limit=10.0] (18.6796, 5.424) -- (18.6796, 5.1541) -- (18.6796, 4.9213);



  \path[draw=black,fill=white,miter limit=10.0,cm={ -1.0,-0.0,0.0,-1.0,(37.3592, 12.1708)}] (17.6212, 6.7469) -- (19.7379, 6.7469) -- (20.2671, 6.3235) -- (20.2671, 5.424) -- (17.0921, 5.424) -- (17.0921, 6.3235) -- cycle;



  \begin{scope}[shift={(-0.0132, 0.0132)}]
    \node[text=black,anchor=south,fit={(0,0) (3.175, 1.5875)}] (text1446) at (18.6796, 5.9796){Пока остались  пиксели в окрестности};



  \end{scope}

\end{tikzpicture}

\caption{Схема алгоритма мягких теневых карт с фильтрацией (PCSS)}
\label{chart:shadow_map_pcss}

\end{figure}

\FloatBarrier
\begin{figure}
\centering
\begin{tikzpicture}[y=1cm, x=1cm, yscale=\globalscale*0.74,xscale=\globalscale*0.74, every node/.style={font=\scriptsize}, every node/.append style={scale=\globalscale}, inner sep=0pt, outer sep=0pt]
  \path[draw=black,miter limit=10.0,dash pattern=on 0.0794cm off 0.0794cm] (13.1233, 19.9708) -- (13.3879, 19.9708) -- (13.6525, 19.9713);



  \path[draw=black,fill=white,dash pattern=on 0.0794cm off 0.0794cm] (8.89, 22.0813) rectangle (13.1233, 17.8612);



  \path[draw=black,miter limit=10.0,dash pattern=on 0.0794cm off 0.0794cm] (4.2333, 14.6598) -- (4.7625, 14.6465);



  \path[draw=black,fill=white,dash pattern=on 0.0794cm off 0.0794cm] (0.0, 19.9514) rectangle (4.2333, 9.3681);



  \path[draw=black,miter limit=10.0,dash pattern=on 0.0794cm off 0.0794cm] (3.7042, 20.9833) -- (4.7625, 20.9833);



  \path[draw=black,miter limit=10.0] (2.1167, 20.1896) -- (2.1167, 19.6604);



  \path[draw=black,fill=white,rounded corners=0.4445cm] (0.5292, 21.7771) rectangle (3.7042, 20.1896);



  \begin{scope}[shift={(-0.0132, 0.0132)}]
    \node[text=black,anchor=south,fit={(0,0) (3.175, 1.5875)}] (text5442) at (2.1167, 20.8756){Начало};



  \end{scope}
  \path (4.7625, 22.3062) rectangle (5.7944, 19.6604);



  \path[draw=black,miter limit=10.0] (5.7944, 22.3062) -- (4.7625, 22.3062) -- (4.7625, 19.6604) -- (5.7944, 19.6604);



  \begin{scope}[shift={(-0.0132, 0.0132)}]
    \node[text=black,anchor=south,fit={(0,0) (3, 1.5875)},scale=1] (text5883) at (6.2, 20.8756){Алгоритм теневых карт с фильтрацией (PCF)};



  \end{scope}
  \path[draw=black,miter limit=10.0] (2.1167, 18.0729) -- (2.1167, 17.557);



  \path[draw=black,fill=white] (0.5292, 19.6604) rectangle (3.7042, 18.0729);



  \begin{scope}[shift={(-0.0132, 0.0132)}]
    \node[text=black,anchor=south,fit={(0,0) (3.175, 1.5875)}] (text1439) at (2.1167, 18.759){Инициализировать теневую карту};



  \end{scope}
  \path[draw=black,miter limit=10.0] (2.1167, 15.9695) -- (2.1167, 15.4535);



  \path[draw=black,fill=white] (0.5292, 17.557) rectangle (3.7042, 15.9695);



  \begin{scope}[shift={(-0.0132, 0.0132)}]
    \node[text=black,anchor=south,fit={(0,0) (3.175, 1.5875)}] (text4655) at (2.1167, 16.6688){Инициализировать источник света};



  \end{scope}
  \path[draw=black,miter limit=10.0] (2.1167, 13.866) -- (2.1167, 13.3369);



  \path[draw=black,fill=white] (0.5292, 15.4535) rectangle (3.7042, 13.866);



  \begin{scope}[shift={(-0.0132, 0.0132)}]
    \node[text=black,anchor=south,fit={(0,0) (3.175, 1.5875)}] (text9135) at (2.1167, 14.5521){Инициализировать камеру};



  \end{scope}
  \path[draw=black,miter limit=10.0] (2.1167, 11.7494) -- (2.1167, 11.2202);



  \path[draw=black,fill=white] (0.5292, 13.3369) rectangle (3.7042, 11.7494);



  \begin{scope}[shift={(-0.0132, 0.0132)}]
    \node[text=black,anchor=south,fit={(0,0) (3.175, 1.5875)}] (text9597) at (2.1167, 12.4354){Инициализировать модели};



  \end{scope}
  \path (4.7625, 16.1962) rectangle (5.7944, 13.0969);



  \path[draw=black,miter limit=10.0] (5.7944, 16.1962) -- (4.7625, 16.1962) -- (4.7625, 13.0969) -- (5.7944, 13.0969);



  \begin{scope}[shift={(-0.0132, 0.0132)}]
    \node[text=black,anchor=south,fit={(0,0) (3.175, 1.5875)},scale=0.9] (text121) at (6.35, 14.5521){Подготовительный этап, инициализирующий минимально необходимые данные};



  \end{scope}
  \path[draw=black,miter limit=10.0] (2.1167, 9.6327) -- (2.1167, 9.1017) -- (8.0963, 9.1017) -- (8.0963, 22.352) -- (11.0067, 22.352) -- (11.0067, 21.8035);



  \path[draw=black,fill=white] (0.5292, 11.2202) rectangle (3.7042, 9.6327);



  \begin{scope}[shift={(-0.0132, 0.0132)}]
    \node[text=black,anchor=south,fit={(0,0) (3.175, 1.5875)}] (text1075) at (2.1167, 10.3188){Инициализировать
шейдеры};



  \end{scope}
  \path[draw=black,miter limit=10.0] (11.0067, 20.216) -- (11.0067, 19.6869);



  \path[draw=black,fill=white] (9.4192, 21.8035) rectangle (12.5942, 20.216);



  \begin{scope}[shift={(-0.0132, 0.0132)}]
    \node[text=black,anchor=south,fit={(0,0) (3.175, 1.5875)}] (text8288) at (11.0067, 20.9021){Нарисовать сцену с точки зрения источника света};



  \end{scope}
  \path[draw=black,miter limit=10.0] (11.0067, 18.0994) -- (11.0067, 17.3321);



  \path[draw=black,fill=white] (9.4192, 19.6869) rectangle (12.5942, 18.0994);



  \begin{scope}[shift={(-0.0132, 0.0132)}]
    \node[text=black,anchor=south,fit={(0,0) (3.175, 1.5875)}] (text6434) at (11.0067, 18.7854){Заполнить теневую карту значениями глубины};



  \end{scope}
  \path (13.6525, 21.2942) rectangle (14.6844, 18.6484);



  \path[draw=black,miter limit=10.0] (14.6844, 21.2942) -- (13.6525, 21.2942) -- (13.6525, 18.6484) -- (14.6844, 18.6484);



  \begin{scope}[shift={(-0.0132, 0.0132)}]
    \node[text=black,anchor=south,fit={(0,0) (2, 1.5875)} ] (text9891) at (14.75, 19.8702){Этап заполнения теневой карты};



  \end{scope}
  \path[draw=black,miter limit=10.0] (11.0067, 16.0091) -- (11.0067, 15.48);



  \path[draw=black,fill=white,miter limit=10.0] (9.9483, 17.3321) -- (12.065, 17.3321) -- (12.5942, 16.9087) -- (12.5942, 16.0091) -- (9.4192, 16.0091) -- (9.4192, 16.9087) -- cycle;



  \begin{scope}[shift={(-0.0132, 0.0132)}]
    \node[text=black,anchor=south,fit={(0,0) (3.175, 1.5875)}] (text5014) at (11.0067, 16.5629){Обработка N пикселей};



  \end{scope}
  \path[draw=black,miter limit=10.0,dash pattern=on 0.0794cm off 0.0794cm] (12.5942, 14.6862) -- (13.3879, 14.6862);



  \path[draw=black,miter limit=10.0] (11.0067, 13.8925) -- (11.0067, 13.3633);



  \path[draw=black,fill=white] (9.4192, 15.48) rectangle (12.5942, 13.8925);



  \begin{scope}[shift={(-0.0132, 0.0132)}]
    \node[text=black,anchor=south,fit={(0,0) (3.175, 1.5875)},scale=0.8] (text8939) at (11.0067, 14.5785){Преобразовать координаты фрагмента в пространство света};



  \end{scope}
  \path (13.3879, 15.4932) rectangle (14.4463, 13.8792);



  \path[draw=black,miter limit=10.0] (14.4463, 15.4932) -- (13.3879, 15.4932) -- (13.3879, 13.8792) -- (14.4463, 13.8792);



  \begin{scope}[shift={(-0.0132, 0.0132)}]
    \node[text=black,anchor=south,fit={(0,0) (3.175, 1.5875)} ] (text9548) at (14.5, 14.5785){Далее
d1};



  \end{scope}
  \path[draw=black,fill=white,rounded corners=0.4445cm] (0.5292, 1.9844) rectangle (3.7042, 0.3969);



  \begin{scope}[shift={(-0.0132, 0.0132)}]
    \node[text=black,anchor=south,fit={(0,0) (3.175, 1.5875)}] (text5062) at (2.1167, 1.0848){Конец};



  \end{scope}
  \path[draw=black,miter limit=10.0] (11.0067, 9.9237) -- (11.0067, 9.3946);



  \path[draw=black,fill=white,miter limit=10.0] (9.9483, 11.2466) -- (12.065, 11.2466) -- (12.5942, 10.8233) -- (12.5942, 9.9237) -- (9.4192, 9.9237) -- (9.4192, 10.8233) -- cycle;



  \begin{scope}[shift={(-0.0132, 0.0132)}]
    \node[text=black,anchor=south,fit={(0,0) (3.175, 1.5875)},scale=0.9] (text1678) at (11.0067, 10.4775){Обработка S блокирующих, окаймляющий пикселей};



  \end{scope}
  \path[draw=black,miter limit=10.0] (11.0067, 11.7758) -- (11.0067, 11.2466);



  \path[draw=black,miter limit=10.0,dash pattern=on 0.0794cm off 0.0794cm] (12.5942, 12.5624) -- (13.3879, 12.5587);



  \path[draw=black,fill=white] (9.4192, 13.3633) rectangle (12.5942, 11.7758);



  \begin{scope}[shift={(-0.0132, 0.0132)}]
    \node[text=black,anchor=south,fit={(0,0) (3.175, 1.5875)}] (text1321) at (11.0067, 12.4619){sumBlocker = 0};



  \end{scope}
  \path[draw=black,miter limit=10.0,dash pattern=on 0.0794cm off 0.0794cm] (12.5942, 8.608) -- (13.3879, 8.6117);



  \path[draw=black,miter limit=10.0] (11.0067, 7.8071) -- (11.0067, 7.276);



  \path[draw=black,fill=white] (9.4192, 9.3946) rectangle (12.5942, 7.8071);



  \begin{scope}[shift={(-0.0132, 0.0132)}]
    \node[text=black,anchor=south,fit={(0,0) (3.175, 1.5875)}] (text1824) at (11.0067, 8.4931){Считать соответствующую глубину в окрестности};



  \end{scope}
  \path (13.3879, 9.421) rectangle (14.4463, 7.8071);



  \path[draw=black,miter limit=10.0] (14.4463, 9.421) -- (13.3879, 9.421) -- (13.3879, 7.8071) -- (14.4463, 7.8071);



  \begin{scope}[shift={(-0.0132, 0.0132)}]
    \node[text=black,anchor=south,fit={(0,0) (3.175, 1.5875)} ] (text5006) at (14.5, 8.5196){Далее
d2};



  \end{scope}
  \path (13.3879, 13.3633) rectangle (14.4463, 11.7494);



  \path[draw=black,miter limit=10.0] (14.4463, 13.3633) -- (13.3879, 13.3633) -- (13.3879, 11.7494) -- (14.4463, 11.7494);



  \begin{scope}[shift={(-0.0132, 0.0132)}]
    \node[text=black,anchor=south,fit={(0,0) (3.175, 1.5875)},scale=0.9] (text3773) at (14.5, 12.4619){Количество блокирующих глубин};



  \end{scope}
  \path[draw=black,miter limit=10.0] (12.5942, 6.4823) -- (13.1022, 6.477) -- (13.1022, 5.3544);



  \path[draw=black,fill=black,miter limit=10.0] (13.1022, 5.2155) -- (13.0096, 5.4007) -- (13.1022, 5.3544) -- (13.1948, 5.4007) -- cycle;



  \begin{scope}[shift={(-0.0132, 0.0132)}]
    \node[text=black,anchor=south,fit={(0,0) (3.175, 1.5875)}] (text4912) at (12.8323, 6.641){Да};



  \end{scope}
  \path[draw=black,miter limit=10.0] (9.4192, 6.4823) -- (8.7842, 6.477) -- (8.7842, 3.0692) -- (11.0067, 3.0692) -- (11.0067, 2.7085);



  \path[draw=black,fill=black,miter limit=10.0] (11.0067, 2.5696) -- (10.9141, 2.7548) -- (11.0067, 2.7085) -- (11.0993, 2.7548) -- cycle;



  \path[draw=black,fill=white,miter limit=10.0] (11.0067, 7.276) -- (12.5942, 6.4823) -- (11.0067, 5.6885) -- (9.4192, 6.4823) -- cycle;



  \begin{scope}[shift={(-0.0132, 0.0132)}]
    \node[text=black,anchor=south,fit={(0,0) (3.175, 1.5875)}] (text3271) at (11.0067, 6.3765){d1 > d2};



  \end{scope}
  \path[draw=black,miter limit=10.0] (13.1022, 3.5983) -- (13.1022, 3.0692) -- (11.0067, 3.0692) -- (11.0067, 2.7085);



  \path[draw=black,fill=black,miter limit=10.0] (11.0067, 2.5696) -- (10.9141, 2.7548) -- (11.0067, 2.7085) -- (11.0993, 2.7548) -- cycle;



  \path[draw=black,fill=white] (11.5094, 5.1858) rectangle (14.6844, 3.5983);



  \begin{scope}[shift={(-0.0132, 0.0132)}]
    \node[text=black,anchor=south,fit={(0,0) (3.175, 1.5875)}] (text3007) at (13.0969, 4.2862){sumBlocker += 1};



  \end{scope}
  \path[draw=black,miter limit=10.0] (11.0067, 1.2171) -- (11.0067, 0.6562) -- (16.0338, 0.6562) -- (16.0338, 22.352) -- (18.6796, 22.352) -- (18.6796, 21.8167);



  \path[draw=black,fill=white,miter limit=10.0,cm={ -1.0,-0.0,0.0,-1.0,(22.0133, 3.7571)}] (9.9483, 2.54) -- (12.065, 2.54) -- (12.5942, 2.1167) -- (12.5942, 1.2171) -- (9.4192, 1.2171) -- (9.4192, 2.1167) -- cycle;



  \begin{scope}[shift={(-0.0132, 0.0132)}]
    \node[text=black,anchor=south,fit={(0,0) (3.175, 1.5875)},scale=0.9] (text4100) at (11.0067, 1.7727){Пока остались  блокирующие пиксели};



  \end{scope}
  \path[draw=black,miter limit=10.0,dash pattern=on 0.0794cm off 0.0794cm] (3.7042, 6.4558) -- (4.7625, 6.4558);



  \path[draw=black,miter limit=10.0] (2.1167, 5.6621) -- (2.1167, 4.6567);



  \path[draw=black,fill=white] (0.5292, 7.2496) rectangle (3.7042, 5.6621);



  \begin{scope}[shift={(-0.0132, 0.0132)}]
    \node[text=black,anchor=south,fit={(0,0) (3.175, 1.5875)}] (text6853) at (2.1167, 6.35){aver /= MxM};



  \end{scope}
  \path (4.7625, 7.2628) rectangle (5.4504, 5.6489);



  \path[draw=black,miter limit=10.0] (5.4504, 7.2628) -- (4.7625, 7.2628) -- (4.7625, 5.6489) -- (5.4504, 5.6489);



  \begin{scope}[shift={(-0.0132, 0.0132)}]
    \node[text=black,anchor=south,fit={(0,0) (3.175, 1.5875)} ] (text9927) at (6.5, 6.35){Среднее арифметическое значение в диапазоне [0, 1]};



  \end{scope}
  \path (4.7625, 4.9213) rectangle (5.4504, 2.7781);



  \path[draw=black,miter limit=10.0] (5.4504, 4.9213) -- (4.7625, 4.9213) -- (4.7625, 2.7781) -- (5.4504, 2.7781);



  \begin{scope}[shift={(-0.0132, 0.0132)}]
    \node[text=black,anchor=south,fit={(0,0) (4.5, 1.5875)},scale=0.65] (text9191) at (6.5, 3.7571){Освещение применяется с учетом значения aver, на которое умножается диффузная и спекулярная состовляющие};



  \end{scope}
  \path[draw=black,miter limit=10.0,dash pattern=on 0.0794cm off 0.0794cm] (3.7042, 3.8629) -- (4.2333, 3.8629) -- (4.7625, 3.8523);



  \path[draw=black,miter limit=10.0] (2.1167, 3.0692) -- (2.1167, 1.9844);



  \path[draw=black,fill=white] (0.5292, 4.6567) rectangle (3.7042, 3.0692);



  \path[draw=black,miter limit=10.0] (0.8467, 4.6567) -- (0.8467, 3.0692)(3.3867, 4.6567) -- (3.3867, 3.0692);



  \begin{scope}[shift={(-0.0132, 0.0132)}]
    \node[text=black,anchor=south,fit={(0,0) (3.175, 1.5875)}] (text9131) at (2.1167, 3.7571){Применить\\полное\\освещение};



  \end{scope}
  \path[draw=black,miter limit=10.0] (18.6796, 3.5983) -- (18.6796, 0.127) -- (8.0963, 0.127) -- (8.0963, 7.7999) -- (2.1167, 7.7999) -- (2.1167, 7.2496);



  \path[draw=black,fill=white,miter limit=10.0,cm={ -1.0,-0.0,0.0,-1.0,(37.3592, 8.5196)}] (17.6212, 4.9213) -- (19.7379, 4.9213) -- (20.2671, 4.4979) -- (20.2671, 3.5983) -- (17.0921, 3.5983) -- (17.0921, 4.4979) -- cycle;



  \begin{scope}[shift={(-0.0132, 0.0132)}]
    \node[text=black,anchor=south,fit={(0,0) (3.175, 1.5875)}] (text5500) at (18.6796, 4.154){Пока остались необработанные пиксели};



  \end{scope}
  \path[draw=black,miter limit=10.0,dash pattern=on 0.0794cm off 0.0794cm] (20.2671, 21.023) -- (20.7963, 21.023);



  \path[draw=black,miter limit=10.0] (18.6796, 20.2292) -- (18.6796, 19.6869);



  \path[draw=black,fill=white] (17.0921, 21.8167) rectangle (20.2671, 20.2292);



  \begin{scope}[shift={(-0.0132, 0.0132)}]
    \node[text=black,anchor=south,fit={(0,0) (3.175, 1.5875)}] (text8144) at (18.6796, 20.9285){bd = \(\frac{sumBlocker}{S}\)};



  \end{scope}
  \path (20.7963, 21.83) rectangle (21.4842, 20.216);



  \path[draw=black,miter limit=10.0] (21.4842, 21.83) -- (20.7963, 21.83) -- (20.7963, 20.216) -- (21.4842, 20.216);



  \begin{scope}[shift={(-0.0132, 0.0132)}]
    \node[text=black,anchor=south,fit={(0,0) (3.175, 1.5875)},scale=0.8] (text8957) at (22, 20.9285){Среднее арифметическое значение в диапазоне [0, 1]};



  \end{scope}
  \path[draw=black,miter limit=10.0,dash pattern=on 0.0794cm off 0.0794cm] (20.2671, 18.8931) -- (20.7963, 18.8931);



  \path[draw=black,miter limit=10.0] (18.6796, 18.0994) -- (18.6796, 17.8329) -- (18.6796, 17.5966);



  \path[draw=black,fill=white] (17.0921, 19.6869) rectangle (20.2671, 18.0994);



  \begin{scope}[shift={(-0.0132, 0.0132)}]
    \node[text=black,anchor=south,fit={(0,0) (3.175, 1.5875)}] (text9770) at (18.6796, 18.7854){$penumbraSize = \frac{d1 - bd}{bd}$};



  \end{scope}
  \path (20.7963, 19.7001) rectangle (21.4842, 18.0861);



  \path[draw=black,miter limit=10.0] (21.4842, 19.7001) -- (20.7963, 19.7001) -- (20.7963, 18.0861) -- (21.4842, 18.0861);



  \begin{scope}[shift={(-0.0132, 0.0132)}]
    \node[text=black,anchor=south,fit={(0,0) (3.175, 1.5875)},scale=0.75] (text7127) at (22, 18.7854){Расчет размера полутени};



  \end{scope}
  \path[draw=black,miter limit=10.0] (18.6796, 14.1552) -- (18.6796, 13.8853) -- (18.6796, 13.6279);



  \path[draw=black,fill=white,miter limit=10.0] (17.6212, 15.4781) -- (19.7379, 15.4781) -- (20.2671, 15.0548) -- (20.2671, 14.1552) -- (17.0921, 14.1552) -- (17.0921, 15.0548) -- cycle;



  \begin{scope}[shift={(-0.0132, 0.0132)}]
    \node[text=black,anchor=south,fit={(0,0) (3.175, 1.5875)},scale=0.9] (text334) at (18.6796, 14.7108){Обработка MxM окаймляющий пикселей};



  \end{scope}
  \path[draw=black,miter limit=10.0] (18.6796, 16.0091) -- (18.6796, 15.4781);



  \path[draw=black,fill=white] (17.0921, 17.5966) rectangle (20.2671, 16.0091);



  \begin{scope}[shift={(-0.0132, 0.0132)}]
    \node[text=black,anchor=south,fit={(0,0) (3.175, 1.5875)}] (text9090) at (18.6796, 16.6952){aver = 0};



  \end{scope}
  \path[draw=black,miter limit=10.0,dash pattern=on 0.0794cm off 0.0794cm] (20.2671, 12.8341) -- (20.7963, 12.8341);



  \path[draw=black,miter limit=10.0] (18.6785, 12.0404) -- (18.6785, 11.5112);



  \path[draw=black,fill=white] (17.0921, 13.6279) rectangle (20.2671, 12.0404);



  \begin{scope}[shift={(-0.0132, 0.0132)}]
    \node[text=black,anchor=south,fit={(0,0) (3.175, 1.5875)},scale=0.8] (text6779) at (18.6796, 12.7265){Считать соответствующую глубину в окрестности со сдвигом по шуму};



  \end{scope}
  \path (20.7963, 13.6411) rectangle (21.4842, 12.0272);



  \path[draw=black,miter limit=10.0] (21.4842, 13.6411) -- (20.7963, 13.6411) -- (20.7963, 12.0272) -- (21.4842, 12.0272);



  \begin{scope}[shift={(-0.0132, 0.0132)}]
    \node[text=black,anchor=south,fit={(0,0) (3.175, 1.5875)},scale=0.75] (text1130) at (22, 12.7265){Окрестность масштабируется в penumbraSize раз. Далее d3};



  \end{scope}
  \path[draw=black,miter limit=10.0] (20.266, 10.7175) -- (20.7751, 10.7209) -- (20.7751, 9.5896);



  \path[draw=black,fill=black,miter limit=10.0] (20.7751, 9.4507) -- (20.6825, 9.6359) -- (20.7751, 9.5896) -- (20.8677, 9.6359) -- cycle;



  \begin{scope}[shift={(-0.0132, 0.0132)}]
    \node[text=black,anchor=south,fit={(0,0) (3.175, 1.5875)}] (text198) at (20.5052, 10.8744){Да};



  \end{scope}
  \path[draw=black,miter limit=10.0] (17.091, 10.7175) -- (16.4571, 10.7209) -- (16.4571, 7.3025) -- (18.6796, 7.3025) -- (18.6796, 6.9154);



  \path[draw=black,fill=black,miter limit=10.0] (18.6796, 6.7765) -- (18.587, 6.9617) -- (18.6796, 6.9154) -- (18.7722, 6.9617) -- cycle;



  \path[draw=black,fill=white,miter limit=10.0] (18.6785, 11.5112) -- (20.266, 10.7175) -- (18.6785, 9.9237) -- (17.091, 10.7175) -- cycle;



  \begin{scope}[shift={(-0.0132, 0.0132)}]
    \node[text=black,anchor=south,fit={(0,0) (3.175, 1.5875)}] (text3298) at (18.6796, 10.6098){d1 > d3};



  \end{scope}
  \path[draw=black,miter limit=10.0] (20.7751, 7.8335) -- (20.7751, 7.3025) -- (18.6796, 7.3025) -- (18.6796, 6.9154);



  \path[draw=black,fill=black,miter limit=10.0] (18.6796, 6.7765) -- (18.587, 6.9617) -- (18.6796, 6.9154) -- (18.7722, 6.9617) -- cycle;



  \path[draw=black,fill=white] (19.1812, 9.421) rectangle (22.3562, 7.8335);



  \begin{scope}[shift={(-0.0132, 0.0132)}]
    \node[text=black,anchor=south,fit={(0,0) (3.175, 1.5875)}] (text2209) at (20.7698, 8.5196){aver += 1};



  \end{scope}
  \path[draw=black,miter limit=10.0] (18.6796, 5.424) -- (18.6796, 5.1541) -- (18.6796, 4.9213);



  \path[draw=black,fill=white,miter limit=10.0,cm={ -1.0,-0.0,0.0,-1.0,(37.3592, 12.1708)}] (17.6212, 6.7469) -- (19.7379, 6.7469) -- (20.2671, 6.3235) -- (20.2671, 5.424) -- (17.0921, 5.424) -- (17.0921, 6.3235) -- cycle;



  \begin{scope}[shift={(-0.0132, 0.0132)}]
    \node[text=black,anchor=south,fit={(0,0) (3.175, 1.5875)}] (text1446) at (18.6796, 5.9796){Пока остались  пиксели в окрестности};



  \end{scope}

\end{tikzpicture}

\caption{Схема алгоритма мягких теневых карт с фильтрацией шумом (PCSS-NOISE)}
\label{chart:shadow_map_pcss_noise}

\end{figure}
  
\FloatBarrier

\section*{Вывод}

В данном разделе были представлены схемы алгоритмов модели освещения,
поддерживающей два источника света: точечный и прожекторный -- а также
схемы алгоритмов теневых карт.