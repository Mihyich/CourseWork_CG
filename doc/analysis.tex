\chapter{Аналитическая часть}
\section{Цель и задачи}

\textbf{Цель} -- изучение метода умножения матриц по стандартному алгоритму и
Копперсмита-Винограда.

Задачи:

\begin{itemize}
	\item изучить метод умножения матриц;
	\item разработать алгоритмы умножения матриц: стнадартный, Копперсмит-Виноград и оптимизированный Копперсмит-Виноград;
	\item реализовать разработанные алгоритмы;
	\item выполнить оценку затрат алгоритмов по памяти;
	\item выполнить замеры процессорного времени работы реализаций алгоритмов;
\end{itemize} 

\section{Стандартный алгоритм умножения матриц}

Стандартный алгоритм умножения матриц является классическим методом,
основанным на определении матричного произведения.
Для двух матриц \( A \) размером \( m \times n \) и \( B \) размером \( n \times p \),
результатом их произведения является матрица \( C \) размером \( m \times p \),
элементы которой вычисляются по следующей формуле:

\[
C_{ij} = \sum_{k=1}^{n} A_{ik} \cdot B_{kj}
\]

где \( A_{ik} \) — элемент матрицы \( A \), расположенный в строке \( i \) и столбце \( k \), а \( B_{kj} \) — элемент матрицы \( B \), находящийся в строке \( k \) и столбце \( j \). Таким образом, каждый элемент матрицы \( C \) представляет собой скалярное произведение \( i \)-й строки матрицы \( A \) и \( j \)-го столбца матрицы \( B \).

Основные шаги стандартного алгоритма:

\begin{enumerate}
    \item \textbf{Инициализация}: создается результирующая матрица \( C \) размером \( m \times p \), все элементы которой изначально равны нулю.
    \item \textbf{Произведение}: для каждого элемента \( C_{ij} \) выполняется сумма произведений элементов соответствующей строки матрицы \( A \) и столбца матрицы \( B \).
    \item \textbf{Заполнение матрицы}: по мере вычисления элементов результирующая матрица \( C \) заполняется.
\end{enumerate}

\section{Алгоритм Копперсмита–Винограда}

Алгоритм Копперсмита–Винограда является одним из продвинутых методов умножения матриц,
который улучшает асимптотическую сложность по сравнению со стандартным алгоритмом.
Хотя алгоритм Копперсмита–Винограда имеет теоретическое преимущество,
его практическое использование ограничено, поскольку выигрыш по времени появляется только при очень больших размерах матриц.

Для двух матриц \( A \) размером \( m \times n \) и \( B \) размером \( n \times p \), произведение матриц \( C = A \times B \) можно выразить следующим образом. Основной идеей алгоритма является минимизация количества необходимых операций умножения за счет использования специальных разложений и симметрий.

\begin{enumerate}
    \item \textbf{Подготовительный этап}: вычисляются промежуточные суммы для строк матрицы \( A \) и столбцов матрицы \( B \), что позволяет сократить количество операций умножения на следующем этапе.
    
    Для каждой строки матрицы \( A \) и каждого столбца матрицы \( B \) вычисляются следующие величины:
    \[
    \text{row\_sum}_i = \sum_{k=1}^{\lfloor n/2 \rfloor} A_{ik} \cdot A_{i, n-k+1}
    \]
    \[
    \text{col\_sum}_j = \sum_{k=1}^{\lfloor n/2 \rfloor} B_{kj} \cdot B_{n-k+1, j}
    \]
    
    \item \textbf{Основной этап}: основной этап алгоритма заключается в вычислении элементов результирующей матрицы \( C \) с использованием вычисленных ранее промежуточных сумм:
    \[
    C_{ij} = - \text{row\_sum}_i - \text{col\_sum}_j + \sum_{k=1}^{\lfloor n/2 \rfloor} 
    (A_{ik} + B_{n-k+1,j}) \cdot (A_{i,n-k+1} + B_{k,j})
    \]
    Если \( n \) нечётно, то последний элемент вычисляется по стандартному правилу умножения матриц.

    \item \textbf{Окончательная коррекция}: если \( n \) нечётное, добавляется коррекция последнего элемента по следующей формуле:
    \[
    C_{ij} = C_{ij} + A_{i, \lceil n/2 \rceil} \cdot B_{\lceil n/2 \rceil, j}
    \]
\end{enumerate}

\section*{Вывод}
В данном разделе были теоретически разобраны алгоритмы умножения матриц.

