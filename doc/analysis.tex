\chapter{Аналитический раздел}

\section{Модель освещения}

В этой части раздела описаны основные идеи Ламбертового освещения~\cite{OpenGL_DevidVolf}.

\subsection*{Диффузное освещение}

Модель описывает равномерное рассеивание света от поверхности во всех направлениях.
Яркость поверхности зависит от угла падения света:

\begin{equation}
    \label{equ:diffuse_lambert}
    I_{\text{diff}} = \max
    \begin{cases}
        \vec{L} \cdot \vec{N} \\
        0
    \end{cases},
\end{equation}

\noindent где \( \vec{L} \) — направление от фрагмента к источнику света,
а \( \vec{N} \) — нормаль к поверхности.

\subsection*{Уменьшение интенсивности с расстоянием}

Интенсивность света уменьшается с увеличением расстояния между источником и фрагментом:

\begin{equation}
    \label{equ:atteunation_lambert}
    A = \max
    \begin{cases}
        1 - \frac{d}{R} \\
        0
    \end{cases},
\end{equation}

\noindent где \( d \) — расстояние до источника, а \( R \) — радиус действия света.

\subsection*{Итоговая интенсивность и цвет освещения}

Цвет и интенсивность освещения зависят от мощности источника (\(Intensity\)), его цвета (\(Color\)), а также от угла падения \(I_{\text{diff}}\) -- (\ref{equ:diffuse_lambert}) и расстояния \(A\) -- (\ref{equ:atteunation_lambert}):

\begin{equation}
    \label{equ:std_lambert}
    \text{LightColor} = \text{Color} \cdot \text{Intensity} \cdot A \cdot I_{\text{diff}},
\end{equation}

\section{Типы источников света}

\subsection{Точечный источник света}

Точечный источник света \enquote{Point Light} испускает свет равномерно во всех направлениях из одной точки. 

Такой источник света описывается следующими параматрами:

\begin{itemize}
    \item[-] \textbf{позиция}: \(\vec{P}\) -- координаты источника света,
    \item[-] \textbf{радиус действия}: \(R\) -- ограничивает зону влияния света,
    \item[-] \textbf{цвет}: определяет оттенок света,
    \item[-] \textbf{интенсивность}: задаёт силу света.
\end{itemize}

Расчет освещения для точечного источника света вычисляется по формулам~(\ref{equ:diffuse_lambert}~--~\ref{equ:std_lambert}).

\subsection{Прожекторный источник света}

Прожекторный источник света \enquote{Spot Light} испускает свет в определённом направлении,
ограниченном конусом.

Такой источник света описывается следующими параматрами:

\begin{itemize}
    \item[-] \textbf{позиция}: \(\vec{P}\) — координаты источника света,
    \item[-] \textbf{направление}: \(\vec{D}\) — вектор направления света,
    \item[-] \textbf{углы конуса}: \(\theta_{\text{inner}}\) и \(\theta_{\text{outer}}\) — внутренний и внешний углы, задающие форму конуса,
    \item[-] \textbf{радиус действия}: \(R\) — зона действия света,
    \item[-] \textbf{цвет}: определяет оттенок света,
    \item[-] \textbf{интенсивность}: сила света внутри конуса.
\end{itemize}

Расчет освещения для прожекторного источника представлен ниже.

\subsection*{Диффузное освещение}

Яркость поверхности, зависящая от угла падения света, вычисляется аналогично
алгоритму освещения для точечного источника света, по формуле~(\ref{equ:diffuse_lambert}).

\subsection*{Уменьшение интенсивности с расстоянием}

Однотипно интенсивность света уменьшается с увеличением расстояния между источником и фрагментом
по формуле~(\ref{equ:atteunation_lambert}).

\subsection*{Формирование направления интенсивности света прожектора}

Основной шаг алгоритма, ограничивающий большую часть всенаправленного
рассеивания точечного источника света и оставляющий только освещенную
прожектором область, за счет учета дополнительного коэффициента:

\begin{equation}
    \label{equ:spot_effect_lambert}
    \text{spotEffect} = \vec{D} \cdot (-\vec{L}),
\end{equation}

\noindent где \( \vec{L} \) — направление от фрагмента к источнику света.

\subsection*{Смягчение рассеивания обода прожекторного луча}

Для большей реалистичности освещения прожекторным источником света,
вводится учет рассеивания обода освещаемой области. Таким образом,
интенсивность света уменьшается не только с расстоянием относительно
позиции источника света, но и относительно отстояния от направления
прожекторного света:

\begin{equation}
    \label{equ:spot_intensity_lambert}
    \text{spotIntensity} = \min
    \begin{cases}
        \max
        \begin{cases}
            \frac{\text{spotEffect} - \cos(\theta_{\text{outer}})}{\cos(\theta_{\text{inner}}) - \cos(\theta_{\text{outer}})} \\
            0
        \end{cases} \\
        1
    \end{cases}
\end{equation}

Итоговое освещение учитывает затухание и ограничение углом:

\begin{equation}
    \label{equ:spot_lambert}
    \text{LightColor} = \text{Color} \cdot \text{Intensity} \cdot \text{spotIntensity} \cdot A \cdot I_{\text{diff}}.
\end{equation}

\section{Алгоритм теневых карт}

Эта концепция была представлена Лэнсом Уильямсом в 1978 году
в статье под названием
«Отбрасывание изогнутых теней на изогнутые поверхности»~\cite{history}.

Тени создаются путём проверки того, виден ли пиксель из источника света,
за счет сравнения пикселя с z-буфером или глубинным изображением источника света,
сохранённым в виде текстуры~\cite{OpenGL_DevidVolf}.

Его идея заключается в том, чтобы сначала определить,
какие области сцены освещены с точки зрения источника света,
а затем использовать эту информацию для вычисления теней
при рисовании сцены с точки зрения камеры.

\subsection*{Рисование сцены с точки зрения источника света}

На этом этапе строится \textit{теневая карта} — текстура глубины,
которая хранит расстояния от источника света до ближайших объектов сцены в его проекции.

Проекция сцены выполняется с точки зрения источника света, используя матрицу проекции.
Результат сохраняется в текстуру глубины \enquote{shadow map}.

\subsection*{Преобразование координат фрагмента в пространство источника света}

Каждый фрагмент сцены, видимый с точки зрения камеры,
должен быть проверен на предмет того, находится ли он в тени.
Для этого его мировые координаты преобразуются в пространство источника света:

\begin{equation}
    \label{equ:projected_coords}
    \text{projCoords}_{xy} = \frac{\text{LightSpacePos}_{xyz}}{\text{LightSpacePos}_w}.
\end{equation}

Эти координаты нормализуются в диапазон \([0, 1]\):

\begin{equation}
    \label{equ:normilized_coords}
    \text{projCoords}_{xy} = \text{projCoords}_{xy} \cdot 0,5 + 0,5.
\end{equation}

\subsection*{Сравнение глубины фрагмента с теневой картой}

После преобразования проверяется,
является ли текущий фрагмент ближе к источнику света,
чем ближайшая точка, записанная в теневой карте:

\begin{equation}
    \label{equ:depth_check}
    \text{shadow} = 
    \begin{cases}
        1, & \text{если $currentDepth < closestDepth + bias$ }\\
        0, & \text{иначе}
    \end{cases},
\end{equation}

\noindent где:

\begin{itemize}
    \item[-] \(\text{closestDepth}\) — глубина, сохранённая в теневой карте,
    \item[-] \(\text{currentDepth}\) — глубина текущего фрагмента,
    \item[-] \(\text{bias}\) — небольшое смещение для устранения артефактов (погрешностей из-за ошибок вычислений).
\end{itemize}

\(\text{closestDepth}\) -- вычисляется по формуле:

\begin{equation}
    \label{equ:texture_std}
    \text{closestDepth} = texture(\text{shadowMap}, \text{projCoords}_{xy}),
\end{equation}

\noindent где $texture$ - функция чтения из теневой карты по координатам.

Если фрагмент находится дальше, чем значение из теневой карты,
то он попадает в тень.

\subsection*{Учет теней при расчёте освещения}

Результирующий свет для фрагмента корректируется с учётом теней:

\begin{equation}
    \label{equ:shadow_map_lambert}
    \text{lighting} = \text{ambient} + (1 - \text{shadow}) \cdot \text{diffuse}.
\end{equation}

Таким образом, если фрагмент в тени, то только амбиентный свет влияет на его цвет.

\section{Алгоритм теневых карт с фильтрацией}

Алгоритм \textit{Percentage-Closer Filtering} (PCF) --
это метод улучшения теней при использовании теневых карт.
Он помогает смягчить резкие границы теней,
возникающие из-за дискретного характера теневой карты.
В основе PCF лежит использование размытия глубины
путём усреднения результатов нескольких выборок вокруг текущего фрагмента.

\subsection*{Рисование сцены с точки зрения источника света}

Заполнение теневой карты аналогично этапу в стандартном алгоритме.

\subsection*{Преобразование координат фрагмента в пространство источника света}

Координаты текущего фрагмента преобразуются в пространство источника света
аналогично стандартному алгоритму теневых карт по формулам (\ref{equ:projected_coords})
и (\ref{equ:normilized_coords}).

\subsection*{Несколько выборок из теневой карты}

Для применения PCF выполняются несколько выборок глубины
вокруг координаты фрагмента в текстуре теневой карты.
Берутся выборки в соседних точках:

\begin{equation}
    \label{equ:filtering_pcf}
    \text{sampleDepth}_i = \text{texture}(\text{shadowMap}, \text{projCoords}.xy + \text{offset}_i).
\end{equation}

\subsection*{Учет каждой выборки}

Для каждой выборки проверяется, находится ли фрагмент в тени,
аналогично стандартному алгоритму теневых карт по формуле (\ref{equ:depth_check}):

\begin{equation}
    \label{equ:depth_check_pcf}
    \text{shadow}_i = 
    \begin{cases}
        1, & \text{если $currentDepth < sampleDepth_i + bias$} \\
        0, & \text{иначе}
    \end{cases}
    .
\end{equation}

\subsection*{Усреднение теневых значений}

Результирующее значение тени — это среднее значение всех выборок:

\begin{equation}
    \label{equ:aver_factor_pcf}
    \text{shadowFactor} = \frac{1}{N} \sum_{i=1}^{N} \text{shadow}_i,
\end{equation}

\noindent где \(N\) — количество выборок.

\subsection*{Корректировка освещения с учетом PCF}

Окончательное освещение рассчитывается с учётом результата PCF:

\begin{equation}
    \label{equ:pcf_lambert}
    \text{lighting} = \text{ambient} + (1 - \text{shadowFactor}) \cdot \text{diffuse}.
\end{equation}

\section{Алгоритм теневых карт с фильтрацией шумом}

Шумовая фильтрация \enquote{Noise Filtering} используется для смягчения теней и
устранения артефактов из-за разрывов или резких переходов на теневой карте.
Вместо регулярного фильтра (как \enquote{PCF}), в данном подходе добавляются случайные смещения,
что создаёт эффект дезориентированных теней и уменьшает эффект лесенки \enquote{aliasing}.
Эти случайные смещения генерируются с помощью шума, часто с использованием случайных или Гауссовых распределений.

\subsection*{Подготовка данных}

Рисование сцены с точки зрения источника света и
преобразование координат фрагмента в пространство источника света
происходят аналогично стандартному алгоритму теневой карты.

\subsection*{Генерация случайного числа}

\begin{equation}
    \label{equ:rand}
    f(\vec{v}) = fract(\sin(\vec{v} \cdot \vec{c}) \cdot k),
\end{equation}

\noindent где:

\begin{itemize}
    \item[-] $fract(x)$ -- функция, возвращающая дробную часть числа,
    \item[-] $\vec{v}$ -- вектор, для которого вычисляется случайный вектор,
    \item[-] $\vec{c}$ -- вектор-константа, подбирается случайно,
    \item[-] $k$ -- число для нормализации значения, как правило, большое, подбирается случайно.
\end{itemize}

\subsection*{Генерация Гауссового распределения}

Для добавления случайного смещения используется метод Бокса-Мюллера
для генерации чисел с Гауссовым распределением.

\begin{equation}
    \label{equ:Box_Muller_Gauss}
    r = \sqrt{-2 \ln u_1} \cdot \cos(2 \pi u_2),
\end{equation}

\noindent где $u_1$ и $u_2$ -- случайные числа в диапазоне \([0, 1]\),
компоненты вектора, полученного в результате функции (\ref{equ:rand}).

\subsection*{Итоговое смещение}

\begin{equation}
    \label{equ:noise_offest}
    \text{noiseOffset} = (r_1, r_2) \cdot \sigma,
\end{equation}

\noindent где:

\begin{itemize}
    \item[-] $r_1$ и $r_2$ -- независимые случайные числа с Гауссовым распределением,
    \item[-] $\sigma$ -- стандартное отклонение (ширина разброса).
\end{itemize}

\subsection*{Фильтрация}

Этапы алгоритма фильтрации аналогичны шагам в PCF,
за исключением того, что вместо $offest$ берется
переменая $noiseOffset$ для формулы (\ref{equ:filtering_pcf}).

\section{Алгоритм мягких теневых карт с фильтрацией}

Алгоритм PCSS (Percentage-Closer Soft Shadows) --
это улучшенный метод теневого сглаживания,
который моделирует мягкие тени.
В отличие от стандартного PCF,
который использует фиксированный радиус выборок,
PCSS динамически определяет радиус на основе расстояния от источника света и
близлежащих блокирующих объектов.
Это позволяет лучше передавать эффект мягких теней,
как в реальной жизни: чем дальше объект от поверхности,
тем более размытой становится тень.

\subsection*{Подготовка данных}

Рисование сцены с точки зрения источника света и
преобразование координат фрагмента в пространство источника света
происходят аналогично стандартному алгоритму теневой карты.

\subsection*{Поиск блокирующих объектов}

Задача - оценить объекты, которые частично блокируют свет и влияют на размер полутени.

\begin{equation}
    \label{equ:search_blocker}
    avgBlockerDepth = \frac{blockerDepthSum}{numBlockers},
\end{equation}

\noindent где $numBlockers$ -- количество выборок,
а $blockerDepthSum$ вычисляется по формуле:

\begin{equation}
    \label{euq:blocker_depth_sum}
    sum = \sum_{i = 1}^{N}
    \begin{cases}
        1, & \text{если $currentDepth_i < sampleDepth_i + bias$} \\
        0, & \text{иначе}
    \end{cases},
\end{equation}

\noindent где $sampleDepth_i$ вычисляется аналогично алгоритму PCF по формуле (\ref{equ:filtering_pcf}),
за исключением того, что $offset_i$ вычисляется по формуле:

\begin{equation}
    \label{equ:offset_pcss}
    offset_i = sampleDisk(i, N) \cdot texelSize \cdot searchRadius,
\end{equation}

\noindent где:

\begin{itemize}
    \item[-] $texelSize$ -- размер одного элемента в текстуре теней,
    \item[-] $searchRadius$ -- радиус поиска потенциальных блокирующих объектов, 
    \item[-] $sampleDisk(i, N)$ -- генерирация выборки, равномерно распределенной по окружности (\ref{equ:sample_disk}).
\end{itemize}

Формула генерации выборки, равномерно распределенной по окружности:

\begin{equation}
    \label{equ:sample_disk}
    sampleDisk(i, N) = \{\cos(rad); \sin(rad)\},    
\end{equation}

\noindent где

\begin{equation*}
    rad = 2 \pi \cdot\frac{i}{N},
\end{equation*}

\noindent где

\begin{itemize}
    \item[-] $i$ -- индекс очередной выборки,
    \item[-] $N$ -- общеее количество выборок.
\end{itemize}

\subsection*{Оценка размера полутени}

Задача -- определить, насколько велика полутень на основе разницы
глубин между фрагментом и блокирующими объектами.

Видовая глубина фрагмента и блокирующего объекта:

\begin{equation}
    \label{equ:view_depth_object}
    viewLightFragDepth = \frac{P_y}{2 \cdot currentDepth - P_x - 1},
\end{equation}

\begin{equation}
    \label{equ:view_depth_blocker}
    viewLightBlockerDepth = \frac{P_y}{2 \cdot avgBlockerDepth - P_x - 1},
\end{equation}

\noindent где

\[P_x = \frac{n + f}{f - n}\]

\noindent и

\[P_y = -\frac{2 n f}{f - n},\]

\noindent где $n$ -- ближняя проекционная плоскость отсечения,
$f$ -- дальняя проекционная плоскость отсечения.

Оценка размера полутени:

\begin{equation}
    \label{equ:penumbra}
    penumbraSize = \frac{viewLightFragDepth - viewLightBlockerDepth}{viewLightBlockerDepth} \cdot R,
\end{equation}

\noindent где $R$ -- радиус действия источника света.

Чем больше разница между глубинами, тем шире становится полутень.

\subsection*{Применение PCF с изменяемым радиусом}

Задача -- использовать Percentage-Closer Filtering (PCF) для размытия границ теней
в зависимости от размера полутени.

\begin{equation}
    \label{equ:depth_pcss}
    depth_{xy} = texture(shadowMap, projCoords_{xy} + (x,y) \cdot texelSize \cdot fR),
\end{equation}

\noindent где $x \in [1, 2, .. halfSize]$ и $y \in [1, 2, .. halfSize]$, где $halfSize = \frac{samleSize}{2}$,
где $samleSize$ -- количество выборок.

Итоговое значение для тени:

\begin{equation}
    \label{equ:pcf_pcss}
    shadow = \frac{1}{c} \cdot \sum_{x = 1}^{halfSize}\sum_{y = 1}^{halfSize}
    \begin{cases}
        1, & \text{$currentDepth - bias > depth_{xy}$} \\
        0, & \text{иначе}
    \end{cases},
\end{equation}

\noindent где $c = (2 \cdot halfSize + 1)^{2}$

\subsection*{Итоговый расчет тени}

Влияние тени учитывается аналогично формуле (\ref{equ:pcf_lambert}) в алгоритме
стандартной фильтрации PCF, за исключением того, что вместо переменной $shadowFactor$
берется значение вычисленное с учетом полутенени -- $shadow$.

\section{Алгоритм мягких теневых карт с фильтрацией шумом}

Данный алгоритм является гибридным и сожержит в себе идеи из
сразу нескольких: стандартного алгоритма теневых карт,
алгоритма теневых карт с фильтрацией шумом - и является
расширением алгоритма мягких теневых карт (PCSS).

Все этапы алгоритма будут идентичны шагам в алгоритме мягких
теней PCSS, за исколючением того, что в формулах, использующих
$offset_i$, вместо вычисления равномерно расположенной позиции на окружноасти
будет использоваться вариант Гауссового распределения (формулы: \ref{equ:rand} - \ref{equ:noise_offest}).

\section{Параметры алгоритмов теневых карт}

В описаных выше алгоритмах теневых карт выделяются общие параметры:

\begin{enumerate}[label=\arabic*), labelsep=0.5em]
    \item параметры в этапе заполнения теневой карты:
    \begin{itemize}[label=---]
        \item \textbf{view}: матрица вида источника света,
        \item \textbf{projection}: матрица проекции источника света,
        \item \textbf{model}: матрица модели объекта,
    \end{itemize}
    \item параметры в заключительном этапе рисования сцены:
    \begin{itemize}[label=---]
        \item \textbf{view}: матрица вида источника света,
        \item \textbf{projection}: матрица проекции источника света,
        \item \textbf{view}: матрица вида камеры пользователя,
        \item \textbf{projection}: матрица проекции камеры пользователя,
        \item \textbf{model}: матрица модели объекта,
        \item \textbf{shadowBias}: корректирующее смещение.
    \end{itemize}
\end{enumerate}

Помимо приведенных выше параметров для алгоритмов с фильтрацией
используется специфичный параметр \textbf{pcfRadius}.

\section*{Вывод}

В данном разделе была выбрана модель освещения,
были теоретически разобраны алгоритмы теневых карт и
описаны поддерживаемые источники света в сцене,
структурированы параметры источников освещения в
Ламбертовой модели, параметры алгоритмов теневых карт.


